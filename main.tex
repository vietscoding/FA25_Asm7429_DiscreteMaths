\documentclass[12pt, a4paper, twoside]{report} % hoặc book
% Languages & font
\usepackage[english]{babel}

% Layout & page
\usepackage{geometry}
\usepackage{fancyhdr}

% Graphic & tables
\usepackage{graphicx}

% References & Citiations
\usepackage{natbib}

% Tiện ích lặt vặt
\usepackage{lipsum} % Văn bản vô nghĩa

\usepackage{blindtext}
\usepackage{titlesec}
\usepackage{setspace}
\usepackage{amsmath}

\geometry{
    top=1cm,          % Lề trên 1cm, bao gồm header
    bottom=1cm,       % Lề dưới 1cm, bao gồm footer
    left=1.25cm,
    right=1cm,
    headheight=12pt,  % Chiều cao header
    headsep=5mm,      % Khoảng cách từ header đến text
    footskip=8mm,     % Khoảng cách từ text đến footer
    includehead,      % Bao gồm header trong printable area
    includefoot       % Bao gồm footer trong printable area
}
% \usepackage{showframe} % Hiển thị khung lề
\usepackage{fancyhdr} % tùy chỉnh header & footer
\usepackage{layouts}   % Thêm gói layouts để hiển thị kích thước
% \usepackage{fontspec}
% \setmainfont{Calibri}

% References & Citiations
\usepackage{natbib}

\usepackage[hidelinks]{hyperref}

\onehalfspacing   % giãn dòng 1.5

\bibliographystyle{agsm}     % định dạng Harvard (có thể thay đổi)

% Đặt kiểu header/footer
\pagestyle{fancy}
\fancyhf{} % xóa sạch mặc định

% Header: odd/even (tự động lấy Chapter/Section)
\fancyhead[LO]{\leftmark}   % trang lẻ: Chapter
\fancyhead[RE]{\rightmark}  % trang chẵn: Section

% Footer: tác giả + số trang
\fancyfoot[L]{Written by: Doan Quoc Viet - BD00536} 
\fancyfoot[R]{\thepage}

% Làm Chapter và Section hiện đúng trong header
\renewcommand{\chaptermark}[1]{%
  \markboth{Chapter \thechapter: #1}{}} % Đặt \rightmark rỗng khi bắt đầu chương mới
\renewcommand{\sectionmark}[1]{%
  \markright{\thesection. #1}} % Cập nhật \rightmark khi có Section mới



\begin{document}
% Dùng title case cho các chapter và tiêu đề sau chapter (1.1)
% Không viết hoa: mạo từ (a, an, the), giới từ (in, of,...), liên từ đồng đẳng (and, or, but)
% Không chuyển động từ của tiêu đề sang danh động từ, trừ khi brief yêu cầu

% Hiển thị sơ đồ bố cục chi tiết (chỉ chạy một lần để xem)
%\currentpage
%\pagedesign

\tableofcontents 

\newpage


\section*{Introduction}
Mathematics is the foundation of the fields of software engineering and information technology (IT), where discrete mathematics plays an important role in solving complex computational problems. Discrete mathematics, which includes fields such as set theory, graph theory, Boolean algebra, and abstract algebra, equips software engineers with the tools to design efficient algorithms, model data structures, and ensure robust system performance. For example, concepts from discrete mathematics are integral to database query optimization, cryptographic protocols, network routing algorithms, and logic circuit design, enabling practical solutions to real-world challenges (Rosen, 2018).

The application of discrete mathematics in software engineering offers significant benefits. It allows developers to create optimized algorithms with improved time and space complexity, as seen in sorting and searching techniques. Furthermore, discrete structures such as graphs facilitate the modeling of relationships in social networks or communication systems. At the same time, Boolean algebra supports the development of reliable digital circuits, and abstract algebra helps understand symmetries in computational problems (Epp, 2010). By mastering these concepts, engineers can solve complex problems of accuracy and scalability. This assignment explores the practical applications of discrete mathematics in software engineering through a structured analysis organized into four chapters.

% Tham chiếu chéo

\begin{itemize}
  \item Chapter 1 focuses on set theory and its relevance to software engineering \hyperref[chap:LO1]{(LO1)}.
  \begin{itemize}
    \item In Section 1.1, operations on algebraic sets will be applied to solve a problem \hyperref[sec:P1]{(P1)}.
    \item Section 1.2 determines the cardinality of a multiset by factoring it, expressing the result as individual multisets \hyperref[sec:P1]{(P2)}.
    \item Section 1.3 discusses finding the inverse of a function using appropriate mathematical techniques \hyperref[sec:M1]{(M1)}.
    \item Finally, Section 1.4 develops proof principles for verifying properties of sets using membership tables and subset relations \hyperref[sec:D1]{(D1)}.
  \end{itemize}
  \item Chapter 2 reviews graph theory and its role in modeling mathematical structures \hyperref[chap:LO2]{(LO2)}.
  \begin{itemize}
    \item Section 2.1 models contextual problems with binary trees, analyzing two cases with quantitative and qualitative insights \hyperref[sec:P3]{(P3)}.
    \item Section 2.2 applies Dijkstra's algorithm to compute the length of the shortest path between two vertices in an undirected graph \hyperref[sec:P4]{(P4)}.
    \item Section 2.3 evaluates the existence of Euler and Hamiltonian cycles in an undirected graph \hyperref[sec:M2]{(M2)}.
    \item Finally, Section 2.4 constructs a proof of the Five Color Theorem \hyperref[sec:D4]{(D4)}.
  \end{itemize}
  \item Chapter 3 investigates solutions to problem situations using Boolean algebra (LO3).
  \begin{itemize}
    \item Section 3.1 discusses the use of Boolean algebra to solve binary problems in two diverse real-world domains, including real-world applications (P5).
    \item Section 3.2 generates truth tables and corresponding Boolean equations from application situations (P6).
    \item Section 3.3 simplifies Boolean equations using algebraic methods (M3).
    \item Finally, Section 3.4 designs a complex system using logic gates, such as constructing a circuit to detect divisibility by 3 in binary-coded decimal (D3).
  \end{itemize}
  \item Chapter 4 explores concepts applied in abstract algebra (LO4).
  \begin{itemize}
    \item Section 4.1 describes the distinguishing features of different binary operations performed on the same set (P7).
    \item Section 4.2 determines the order of a group and the order of a subgroup in given examples, including an application of Lagrange's theorem (P8).
    \item Section 4.3 confirms whether a given set with a binary operation forms a group (M4).
    \item Finally, Section 4.4 explores the application of group theory in computer science through a prepared presentation on a related topic, such as cryptography or algorithm design (D4).
  \end{itemize}

\end{itemize}



% All sources referenced in this assignment are documented in the References section.

\chapter{Examine Set Theory and Functions Applicable to Software Engineering (LO1)}

\label{chap:LO1}  % Đặt nhãn để tham chiếu chéo (cross-references)

\section{Perform Algebraic Set Operations in a Formulated Mathematical Problem (P1)}
% \sectionmark{Perform Algebraic Set Operations in a Formulated Mathematical Problem (P1)} 
\label{sec:P1}
In this assignment
\begin{itemize}
  \item the symbol $\overline{9b}$ denotes a two-digit natural number,
  \item the symbol $\overline{12a}$ represents a three-digit natural number.
\end{itemize}

For example, if $a=3$, then $\overline{12a} = 123$.

For each student, replace a and b with specific numerical values to obtain the final result.

According to the requirement above, I need to address problems in this assignment using my ID, which is BD00536. $a$ and $b$ ($a < b$) represents the two largest numbers in my ID, so $a$ = 5 and $b$ = 6.

I will address the issues listed below in the respective subsections in this Section:

\begin{itemize}
  \item Let $A$ and $B$ be two non-empty finite sets. Assume that cardinalities of the sets $A$, $B$, and $A \cap B$ are $\overline{9b}$, $\overline{2a}$, and $a + b$, respectively. Determine the cardinality of set $A \cup B$.
  \item Suppose $|A-B|=\overline{3a}$, $|A \cup B|=\overline {11b}$, and $|A \cap B|=\overline{1a}$. Determine $|B|$.
  \item At a local market, there are $\overline {35b}$ customers. Suppose $\overline {11a}$ have purchased fruits, $\overline {9b}$ have purchased vegetables, $\overline {8a}$ have purchased bakery items, $\overline{4b}$ have purchased both fruits and vegetables, $\overline{3b}$ have purchased both vegetables and bakery items, $\overline{2a}$ have purchased both fruits and bakery items, and $\overline{1a}$ have purchased all three categories. How many customers have not purchased anything?
\end{itemize}

In the following subsections I will address these issues and along the way use a = 5 and b = 6 as the two largest digits in my ID to demonstrate.


\subsection{Determine the cardinality of set $A \cup B$}
\subsubsection*{The Problem:}

  Let $A$ and $B$ be two non-empty finite sets. Assume that cardinalities of the sets $A$, $B$, and $A \cap B$ are $\overline{9b}$, $\overline{2a}$, and $a + b$, respectively. Determine the cardinality of set $A \cup B$.

  To solve this problem, I need to determine the cardinality of the set $A \cup B$, based on the given information about the cardinalities of the sets $A$, $B$, and $A \cap B$.

\subsubsection*{Given:}
  Using \(a = 5\) and \(b = 6\) from my ID (BD00536), the cardinalities are computed as follows:
  \begin{itemize}
    \item The cardinality of set \(A\) is \(\overline{9b} = 96\).
    \item The cardinality of set \(B\) is \(\overline{2a} = 25\).
    \item The cardinality of set \(A \cap B\) is \(a + b = 5 + 6 = 11\).
  \end{itemize}
  I need to find the cardinality of the set \(A \cup B\).

\subsubsection*{Formula:}
  The cardinality of the union of two sets is given by the inclusion-exclusion principle \citep{rosen2019}:
  \[
  |A \cup B| = |A| + |B| - |A \cap B|
  \]

\subsubsection*{Substitution:}
  Substituting the given values into the formula:
  \[
  |A \cup B| = 96 + 25 - 11
  \]

  Calculating step-by-step:
  \begin{enumerate}
    \item 96 + 25 = 121
    \item 121 - 11 = 110
  \end{enumerate}


  Thus, the cardinality of the set \( A \cup B \) is:
  \[
  |A \cup B| = 110
  \]

\subsubsection*{Verification:}
  To ensure the result is correct, I check the conditions:

\begin{itemize}
  \item \( |A \cap B| = 11 \), meaning the intersection \( A \cap B \) has 11 elements.
  \item Since \( |A \cap B| \leq |A| \) and \( |A \cap B| \leq |B| \), we have \( 11 \leq 96 \) and \( 11 \leq 25 \), both of which are satisfied.
  \item The sets \( A \) and \( B \) are non-empty and finite, which aligns with the assumptions.
\end{itemize}


Therefore, there are no contradictions, and the calculation is valid.

\subsubsection*{Conclusion:}
The cardinality of the set \( A \cup B \) is $\boxed{110}$.


\subsection{Determine $| B |$}

\subsubsection*{The Problem:}
  Suppose $|A-B|=\overline{3a}$, $|A \cup B|=\overline {11b}$, and $|A \cap B|=\overline{1a}$. Determine $|B|$.

\subsubsection*{Given:}
  Using \(a = 5\) and \(b = 6\) from my ID (BD00536), and based on the given problem we have the following data:
  \begin{itemize}
    \item $\vert A - B \vert = \overline{3a} = 35$
    \item $\vert A \cup B \vert = \overline{11b} = 116$
    \item $\vert A \cap B \vert = \overline{1a} = 15$
  \end{itemize}
  Let's determine $|B|$

\subsubsection*{Formula:}
  I will use the \textbf{principle of inclusion-exclusion} to determine $|B|$:
    \[
    |A \cup B| = |A| + |B| - |A \cap B| \tag{1}
    \]

\subsubsection*{Substitution:}
  \begin{align*}
    116 = &|A| + |B| - 15 \\
    &|A| + |B| = 131. \tag{2}
  \end{align*}

  For \( |A - B| \):
  \[
  |A - B| = |A| - |A \cap B|.
  \]

  \[
  35 = |A| - 15.
  \]

  \[
  |A| = 35 + 15 = 50.
  \]

  Substitute \( |A| = 50 \) into equation (2):
  \[
  50 + |B| = 131.
  \]

  \[
  |B| = 131 - 50 = 81.
  \]

\subsubsection*{Verification:}
  \begin{itemize}
    \item $|A|=50, |B|=81,|A\cap B|=15$.
    \item Check: $|A \cup B| = |A| + |B| - |A \cap B| = 50 + 81 - 15 = 116$, which matches.
    \item Check: $|A - B| = |A| - |A \cap B| = 50 - 15 = 35$, which matches.
  \end{itemize}

\subsubsection*{Conclusion:}
Final answer: \boxed{81}

\subsection{How many customers have not purchased anything?}

\subsubsection*{The Problem:}
At a local market, there are $\overline {35b}$ customers. Suppose $\overline {11a}$ have purchased fruits, $\overline {9b}$ have purchased vegetables, $\overline {8a}$ have purchased bakery items, $\overline{4b}$ have purchased both fruits and vegetables, $\overline{3b}$ have purchased both vegetables and bakery items, $\overline{2a}$ have purchased both fruits and bakery items, and $\overline{1a}$ have purchased all three categories. How many customers have not purchased anything?
\subsubsection*{Given:}
  Using \(a = 5\) and \(b = 6\) from my ID (BD00536), and based on the given problem we have the following data:

  \begin{itemize}
    \item There are $\overline{35b} = 356$ customers:
    \begin{itemize}
      \item $\overline{11a} = 115$ have purchased fruits.
      \item $\overline{9b} = 96$ have purchased vegetables.
      \item $\overline{8a} = 85$ have purchased bakery items.
      \item $\overline{4b} = 46$ have purchased both fruits and vegetables.
      \item $\overline{3b} = 36$ have purchased both vegetables and bakery items.
      \item $\overline{2a} = 25$ have purchased both fruits and bakery items.
      \item $\overline{1a} = 15$ have purchased all three categories.
    \end{itemize}
    \item Let:
    \begin{itemize}
      \item $U$ be the set of all customers.
      \item $A$ is the set of customers who bought fruits.
      \item $B$ is the set of customers who bought vegetables.
      \item $C$ is the set of customers who bought bakery items.
      \item $E$ is the set of customers who bought nothing.
    \end{itemize}
    \item Based on the given data, we refine the above data to support solving the problem in a simpler way by using the concepts of sets and we will use these facts to find the number of customers who bought nothing:
    \begin{itemize}
      \item $|U| = 356$
      \item $|A| = 115$
      \item $|B| = 96$
      \item $|C| = 85$
      \item $|A \cap B| = 46$
      \item $|B \cap C| = 36$
      \item $|A \cap C| = 25$
      \item $|A \cap B \cap C| = 15$
      \item $|E| = $?
    \end{itemize}
  \end{itemize}
\subsubsection*{Formula:}
  We can get \textbf{the number of customers who bought nothing ($|E|$)} by subtracting \textbf{the number of customers who bought at least one category ($|A \cup B \cup C|$)} from the \textbf{total number of customers ($|U|$)}, which means:
  \[
  |E| = |U| - |A \cup B \cup C| \tag{1}
  \]

  I will use the \textbf{principle of inclusion-exclusion} for three sets to determine the number of customers who bought nothing at the store, which is |E|, by finding $|A \cup B \cup C|$:
    \[
    |A \cup B \cup C| = |A| + |B|+|C| - |A \cap B| - |B \cap C| - |A \cap C| + |A \cap B \cap C| \tag{2}
    \]

\subsubsection*{Substitution:}
  Substitute the data we have in the given part into equation (2), we have:
  \begin{align*}
    (2) \Leftrightarrow |A \cup B \cup C| &= 115 + 96 + 85 - 46 - 36 - 25 + 15 \\
                      &= (115 + 96 + 85) - (46 + 36 + 25) + 15 \\
                      &= 296 - 107 + 15 \\
                      &= 189 + 15 \\
                      &= 204
  \end{align*}

  Substitute $|A \cup B \cup C|$ we just got above and $|U|$ into the equation (1) to find $|E|$, we have:
  \begin{align*}
    |E| &= |U| - |A \cup B \cup C| \\
        &= 356 - 204  \\
        &= 152
  \end{align*}

\subsubsection*{Verification:}
To confirm the union calculation step-by-step:
\begin{itemize}
  \item Sum of individual sets: $115 + 96 + 85 = 296$
  \item Subtract pairwise intersections: $296 - 46 - 36 - 25 = 296 - 107 = 189$
  \item Add back the triple intersection: $189 + 15 = 204$
\end{itemize}
This matches the result I did above, so the number of customers who purchased nothing is indeed 152.
\subsubsection*{Conclusion:}
The number of customers who have not purchased anything is \boxed{152}.
\section{Determine the Cardinality of a Given Bag (Multiset) (P2)}
\label{sec:P2}

\subsection{List the bag of prime factors for each of the provided numbers}

\subsubsection*{The Problem:}
List the bag of prime factors for each of the provided numbers:
\begin{itemize}
  \item \(\overline{1a2} = \overline{152}\)
  \item \(\overline{2b0} = \overline{260}\)
\end{itemize}

\subsubsection*{Given:}
Using \( a = 5 \) and \( b = 6 \) from the ID (BD00536), we have:
\begin{itemize}
  \item \(\overline{1a2} = 152\), since \( a = 5 \).
  \item \(\overline{2b0} = 260\), since \( b = 6 \).
\end{itemize}
We need to find the prime factorization of each number and represent them as bags (multisets), where a bag allows repeated elements, and the multiplicity of each prime factor is the number of times it appears in the factorization.

\subsubsection*{Formula:}
To find the bag of prime factors for a number, we perform prime factorization by dividing the number by the smallest possible prime numbers repeatedly until the quotient is 1 \cite{burton2010}. The bag contains all prime factors, including repetitions, based on their multiplicity in the factorization.

\subsubsection*{Substitution:}
Let’s compute the prime factorization for each number.

\textbf{For \(\overline{1a2} = 152\):}
\begin{itemize}
  \item Divide by 2: \( 152 \div 2 = 76 \)
  \item Divide by 2: \( 76 \div 2 = 38 \)
  \item Divide by 2: \( 38 \div 2 = 19 \)
  \item 19 is a prime number.
\end{itemize}
Thus, the prime factorization of 152 is:
\[
152 = 2^3 \cdot 19
\]
The bag of prime factors for 152 is:
\[
B_1 = \{2, 2, 2, 19\}
\]

\textbf{For \(\overline{2b0} = 260\):}
\begin{itemize}
  \item Divide by 2: \( 260 \div 2 = 130 \)
  \item Divide by 2: \( 130 \div 2 = 65 \)
  \item Divide by 5: \( 65 \div 5 = 13 \)
  \item 13 is a prime number.
\end{itemize}
Thus, the prime factorization of 260 is:
\[
260 = 2^2 \cdot 5 \cdot 13
\]
The bag of prime factors for 260 is:
\[
B_2 = \{2, 2, 5, 13\}
\]

\subsubsection*{Verification:}
To verify:
\begin{itemize}
  \item For 152: \( 2 \cdot 2 \cdot 2 \cdot 19 = 8 \cdot 19 = 152 \). Correct.
  \item For 260: \( 2 \cdot 2 \cdot 5 \cdot 13 = 4 \cdot 5 \cdot 13 = 20 \cdot 13 = 260 \). Correct.
\end{itemize}
The bags are:
\begin{itemize}
  \item \( B_1 = \{2, 2, 2, 19\} \)
  \item \( B_2 = \{2, 2, 5, 13\} \)
\end{itemize}

\subsubsection*{Conclusion:}
The bags of prime factors are:
\begin{itemize}
  \item For \(\overline{1a2} = 152\): \( \{2, 2, 2, 19\} \)
  \item For \(\overline{2b0} = 260\): \( \{2, 2, 5, 13\} \)
\end{itemize}

\subsection{Find the cardinalities}

\subsubsection*{The Problem:}
Find the cardinalities of:
\begin{itemize}
  \item Each of the bags above (\( B_1 \) and \( B_2 \)).
  \item The intersection of the bags (\( B_1 \cap B_2 \)).
  \item The union of the bags (\( B_1 \cup B_2 \)).
  \item The difference of the bags (\( B_1 \setminus B_2 \)).
\end{itemize}

\subsubsection*{Given:}
From the previous subsection:
\begin{itemize}
  \item \( B_1 = \{2, 2, 2, 19\} \)
  \item \( B_2 = \{2, 2, 5, 13\} \)
\end{itemize}

\subsubsection*{Formula:}
For multisets (bags), as defined in \cite{epp2020}:
\begin{itemize}
  \item The \textbf{cardinality} of a bag is the total number of elements, counting repetitions.
  \item The \textbf{intersection} \( B_1 \cap B_2 \) contains the elements common to both bags, with the multiplicity of each element being the minimum of its multiplicities in \( B_1 \) and \( B_2 \).
  \item The \textbf{union} \( B_1 \cup B_2 \) contains all elements from both bags, with the multiplicity of each element being the maximum of its multiplicities in \( B_1 \) and \( B_2 \).
  \item The \textbf{difference} \( B_1 \setminus B_2 \) contains elements in \( B_1 \) that remain after removing elements that appear in \( B_2 \), with multiplicities adjusted accordingly (if an element’s multiplicity in \( B_2 \) is less than in \( B_1 \), the difference includes the remaining occurrences).
  \item The cardinality of a bag \( B \), denoted \( |B| \), is the sum of the multiplicities of its distinct elements.
\end{itemize}
Additionally, the cardinality of the union can be verified using the inclusion-exclusion principle for multisets \cite{rosen2019}:
\[
|B_1 \cup B_2| = |B_1| + |B_2| - |B_1 \cap B_2|
\]

\subsubsection*{Substitution:}
Let’s compute each part.

\textbf{Cardinality of each bag:}
\begin{itemize}
  \item For \( B_1 = \{2, 2, 2, 19\} \):
    \[
    |B_1| = 3 \text{ (for the three 2’s)} + 1 \text{ (for the one 19)} = 4
    \]
  \item For \( B_2 = \{2, 2, 5, 13\} \):
    \[
    |B_2| = 2 \text{ (for the two 2’s)} + 1 \text{ (for the one 5)} + 1 \text{ (for the one 13)} = 4
    \]
\end{itemize}

\textbf{Intersection \( B_1 \cap B_2 \):}
To find the intersection, we take the common elements with the minimum multiplicity for each:
\begin{itemize}
  \item Element 2: Multiplicity in \( B_1 \) is 3, in \( B_2 \) is 2. Minimum is 2. Include \( \{2, 2\} \).
  \item Element 19: Multiplicity in \( B_1 \) is 1, in \( B_2 \) is 0. Minimum is 0. Exclude 19.
  \item Element 5: Multiplicity in \( B_1 \) is 0, in \( B_2 \) is 1. Minimum is 0. Exclude 5.
  \item Element 13: Multiplicity in \( B_1 \) is 0, in \( B_2 \) is 1. Minimum is 0. Exclude 13.
\end{itemize}
Thus:
\[
B_1 \cap B_2 = \{2, 2\}
\]
Cardinality:
\[
|B_1 \cap B_2| = 2
\]

\textbf{Union \( B_1 \cup B_2 \):}
For the union, we take all elements with the maximum multiplicity for each:
\begin{itemize}
  \item Element 2: Multiplicity in \( B_1 \) is 3, in \( B_2 \) is 2. Maximum is 3. Include \( \{2, 2, 2\} \).
  \item Element 19: Multiplicity in \( B_1 \) is 1, in \( B_2 \) is 0. Maximum is 1. Include \( \{19\} \).
  \item Element 5: Multiplicity in \( B_1 \) is 0, in \( B_2 \) is 1. Maximum is 1. Include \( \{5\} \).
  \item Element 13: Multiplicity in \( B_1 \) is 0, in \( B_2 \) is 1. Maximum is 1. Include \( \{13\} \).
\end{itemize}
Thus:
\[
B_1 \cup B_2 = \{2, 2, 2, 19, 5, 13\}
\]
Cardinality:
\[
|B_1 \cup B_2| = 3 \text{ (for 2)} + 1 \text{ (for 19)} + 1 \text{ (for 5)} + 1 \text{ (for 13)} = 6
\]

\textbf{Difference \( B_1 \setminus B_2 \):}
For the difference, we remove from \( B_1 \) the elements that appear in \( B_2 \), respecting multiplicities:
\begin{itemize}
  \item Element 2: Multiplicity in \( B_1 \) is 3, in \( B_2 \) is 2. After removing two 2’s, \( 3 - 2 = 1 \). Include \( \{2\} \).
  \item Element 19: Multiplicity in \( B_1 \) is 1, in \( B_2 \) is 0. Remains 1. Include \( \{19\} \).
  \item Elements 5 and 13: Not in \( B_1 \), so they don’t affect the difference.
\end{itemize}
Thus:
\[
B_1 \setminus B_2 = \{2, 19\}
\]
Cardinality:
\[
|B_1 \setminus B_2| = 1 \text{ (for 2)} + 1 \text{ (for 19)} = 2
\]

\subsubsection*{Verification:}
To verify:
\begin{itemize}
  \item Cardinality of \( B_1 \): Counts \( \{2, 2, 2, 19\} \), which has 4 elements. Correct.
  \item Cardinality of \( B_2 \): Counts \( \{2, 2, 5, 13\} \), which has 4 elements. Correct.
  \item Intersection: \( \{2, 2\} \) has 2 elements. Matches the minimum multiplicities.
  \item Union: \( \{2, 2, 2, 19, 5, 13\} \) has 6 elements. Matches the maximum multiplicities.
  \item Difference: Removing two 2’s from \( B_1 \) leaves one 2 and one 19, so \( \{2, 19\} \) has 2 elements. Correct.
  \item Using the inclusion-exclusion principle \cite{rosen2019}:
    \[
    |B_1 \cup B_2| = |B_1| + |B_2| - |B_1 \cap B_2|
    \]
    \[
    6 = 4 + 4 - 2 = 6
    \]
    This confirms the union’s cardinality.
\end{itemize}

\subsubsection*{Conclusion:}
The cardinalities are:
\begin{itemize}
  \item Cardinality of \( B_1 = \{2, 2, 2, 19\} \): \( |B_1| = 4 \)
  \item Cardinality of \( B_2 = \{2, 2, 5, 13\} \): \( |B_2| = 4 \)
  \item Cardinality of the intersection \( B_1 \cap B_2 = \{2, 2\} \): \( |B_1 \cap B_2| = 2 \)
  \item Cardinality of the union \( B_1 \cup B_2 = \{2, 2, 2, 19, 5, 13\} \): \( |B_1 \cup B_2| = 6 \)
  \item Cardinality of the difference \( B_1 \setminus B_2 = \{2, 19\} \): \( |B_1 \setminus B_2| = 2 \)
\end{itemize}
\[
\boxed{
\begin{array}{l}
|B_1| = 4, \\
|B_2| = 4, \\
|B_1 \cap B_2| = 2, \\
|B_1 \cup B_2| = 6, \\
|B_1 \setminus B_2| = 2
\end{array}
}
\]

\section{Determine the Inverse of a Function Using Appropriate Mathematical Techniques (M1)}
\label{sec:M1}

% \subsection{Ascertain whether the given functions are invertible}

% \subsubsection*{The Problem:}
% Ascertain whether the given functions are invertible. If they are, identify the rule for the inverse function \( f^{-1} \).
% \begin{itemize}
%   \item \( f: \mathbb{R} \rightarrow \mathbb{R} \) with \( f(x) = bx + a \).
%   \item \( f: [-b, +\infty) \rightarrow [0, +\infty) \) with \( f(x) = \sqrt{x + b} \).
% \end{itemize}

% \subsubsection*{Given:}
% Using \( a = 5 \) and \( b = 6 \) from the ID (BD00536), we have:
% \begin{itemize}
%   \item First function: \( f: \mathbb{R} \rightarrow \mathbb{R} \), \( f(x) = 6x + 5 \).
%   \item Second function: \( f: [-6, +\infty) \rightarrow [0, +\infty) \), \( f(x) = \sqrt{x + 6} \).
% \end{itemize}
% To determine if each function is invertible, we must check if it is bijective (injective and surjective) on its given domain and codomain \cite{epp2020}. If invertible, we find the inverse by solving \( y = f(x) \) for \( x \) in terms of \( y \).

% \subsubsection*{Formula:}
% A function \( f: A \rightarrow B \) is invertible if it is bijective \cite{epp2020}:
% \begin{itemize}
%   \item \textbf{Injective} (one-to-one): \( f(x_1) = f(x_2) \implies x_1 = x_2 \).
%   \item \textbf{Surjective} (onto): For every \( y \in B \), there exists \( x \in A \) such that \( f(x) = y \).
% \end{itemize}
% To find the inverse \( f^{-1} \), set \( y = f(x) \), solve for \( x \), and express \( x = f^{-1}(y) \), then swap variables to write \( f^{-1}(x) \).

% \subsubsection*{Substitution:}
% We analyze each function separately.

% \textbf{First function: \( f: \mathbb{R} \rightarrow \mathbb{R} \), \( f(x) = 6x + 5 \):}
% \begin{itemize}
%   \item \textbf{Injectivity}: Suppose \( f(x_1) = f(x_2) \).
%     \[
%     6x_1 + 5 = 6x_2 + 5
%     \]
%     \[
%     6x_1 = 6x_2 \implies x_1 = x_2
%     \]
%     Since the coefficient of \( x \) is non-zero (\( b = 6 \neq 0 \)), the function is injective.
%   \item \textbf{Surjectivity}: For any \( y \in \mathbb{R} \), solve \( y = f(x) = 6x + 5 \):
%     \[
%     y = 6x + 5 \implies 6x = y - 5 \implies x = \frac{y - 5}{6}
%     \]
%     Since \( y \in \mathbb{R} \), \( x \in \mathbb{R} \), so there exists an \( x \) for every \( y \), making \( f \) surjective.
%   \item Since \( f \) is bijective, it is invertible. To find the inverse, set \( y = f(x) \):
%     \[
%     y = 6x + 5 \implies y - 5 = 6x \implies x = \frac{y - 5}{6}
%     \]
%     Thus, the inverse is:
%     \[
%     f^{-1}(y) = \frac{y - 5}{6}
%     \]
%     Swapping variables:
%     \[
%     f^{-1}(x) = \frac{x - 5}{6}
%     \]
% \end{itemize}

% \textbf{Second function: \( f: [-6, +\infty) \rightarrow [0, +\infty) \), \( f(x) = \sqrt{x + 6} \):}
% \begin{itemize}
%   \item \textbf{Injectivity}: Suppose \( f(x_1) = f(x_2) \).
%     \[
%     \sqrt{x_1 + 6} = \sqrt{x_2 + 6}
%     \]
%     Since the square root function is one-to-one on \( [0, +\infty) \), square both sides:
%     \[
%     x_1 + 6 = x_2 + 6 \implies x_1 = x_2
%     \]
%     Thus, \( f \) is injective.
%   \item \textbf{Surjectivity}: For any \( y \in [0, +\infty) \), solve \( y = f(x) = \sqrt{x + 6} \):
%     \[
%     y = \sqrt{x + 6} \implies y^2 = x + 6 \implies x = y^2 - 6
%     \]
%     For \( x \in [-6, +\infty) \), we need \( y^2 - 6 \geq -6 \), which simplifies to \( y^2 \geq 0 \), true for all \( y \in \mathbb{R} \). Since \( y \geq 0 \), \( x = y^2 - 6 \) is defined and in the domain, so \( f \) is surjective.
%   \item Since \( f \) is bijective, it is invertible. To find the inverse, from the above:
%     \[
%     y = \sqrt{x + 6} \implies y^2 = x + 6 \implies x = y^2 - 6
%     \]
%     Thus:
%     \[
%     f^{-1}(y) = y^2 - 6
%     \]
%     Swapping variables, with the domain of \( f^{-1} \) being the codomain of \( f \), \( [0, +\infty) \):
%     \[
%     f^{-1}(x) = x^2 - 6, \quad x \geq 0
%     \]
% \end{itemize}

% \subsubsection*{Verification:}
% \begin{itemize}
%   \item \textbf{First function}: Verify \( f(f^{-1}(x)) = x \):
%     \[
%     f(f^{-1}(x)) = f\left( \frac{x - 5}{6} \right) = 6 \cdot \frac{x - 5}{6} + 5 = (x - 5) + 5 = x
%     \]
%     Verify \( f^{-1}(f(x)) = x \):
%     \[
%     f^{-1}(f(x)) = f^{-1}(6x + 5) = \frac{(6x + 5) - 5}{6} = \frac{6x}{6} = x
%     \]
%     Both compositions confirm the inverse.
%   \item \textbf{Second function}: Verify \( f(f^{-1}(x)) = x \) for \( x \geq 0 \):
%     \[
%     f(f^{-1}(x)) = f(x^2 - 6) = \sqrt{(x^2 - 6) + 6} = \sqrt{x^2} = x \quad (\text{since } x \geq 0)
%     \]
%     Verify \( f^{-1}(f(x)) = x \) for \( x \geq -6 \):
%     \[
%     f^{-1}(f(x)) = f^{-1}(\sqrt{x + 6}) = (\sqrt{x + 6})^2 - 6 = (x + 6) - 6 = x
%     \]
%     Both compositions confirm the inverse.
% \end{itemize}

% \subsubsection*{Conclusion:}
% Both functions are invertible:
% \begin{itemize}
%   \item For \( f(x) = 6x + 5 \), the inverse is:
%     \[
%     f^{-1}(x) = \frac{x - 5}{6}
%     \]
%   \item For \( f(x) = \sqrt{x + 6} \), the inverse is:
%     \[
%     f^{-1}(x) = x^2 - 6, \quad x \geq 0
%     \]
% \end{itemize}
% \[
% \boxed{
% \begin{array}{l}
% f(x) = 6x + 5: \quad f^{-1}(x) = \frac{x - 5}{6} \\
% f(x) = \sqrt{x + 6}: \quad f^{-1}(x) = x^2 - 6, \quad x \geq 0
% \end{array}
% }
% \]

% \subsection{Find \( g \circ f \)}

% \subsubsection*{The Problem:}
% Let \( f, g: \mathbb{R} \rightarrow \mathbb{R} \) be defined as \( f(x) = \begin{cases} 2x + a, & x < 0 \\ x^3 + b, & x \geq 0 \end{cases} \) and \( g(x) = bx - a \). Find \( g \circ f \).

% \subsubsection*{Given:}
% Using \( a = 5 \) and \( b = 6 \), we have:
% \begin{itemize}
%   \item \( f(x) = \begin{cases} 2x + 5, & x < 0 \\ x^3 + 6, & x \geq 0 \end{cases} \)
%   \item \( g(x) = 6x - 5 \)
% \end{itemize}
% The composition \( g \circ f \) is defined as \( (g \circ f)(x) = g(f(x)) \).

% \subsubsection*{Formula:}
% For functions \( f \) and \( g \), the composition is \( (g \circ f)(x) = g(f(x)) \), where we apply \( f \) first, then \( g \) to the result \cite{rosen2019}. Since \( f \) is piecewise, we compute \( g(f(x)) \) for each case of \( f \).

% \subsubsection*{Substitution:}
% Since \( f \) is defined piecewise, we compute \( g \circ f \) for each case.

% \textbf{Case 1: \( x < 0 \)}:
% \[
% f(x) = 2x + 5
% \]
% \[
% (g \circ f)(x) = g(f(x)) = g(2x + 5) = 6(2x + 5) - 5 = 12x + 30 - 5 = 12x + 25
% \]

% \textbf{Case 2: \( x \geq 0 \)}:
% \[
% f(x) = x^3 + 6
% \]
% \[
% (g \circ f)(x) = g(f(x)) = g(x^3 + 6) = 6(x^3 + 6) - 5 = 6x^3 + 36 - 5 = 6x^3 + 31
% \]

% Thus, the composition is:
% \[
% (g \circ f)(x) = \begin{cases} 
% 12x + 25, & x < 0 \\
% 6x^3 + 31, & x \geq 0 
% \end{cases}
% \]

% \subsubsection*{Verification:}
% Test the composition at boundary and interior points:
% \begin{itemize}
%   \item For \( x = -1 \) (Case 1):
%     \[
%     f(-1) = 2(-1) + 5 = -2 + 5 = 3, \quad g(f(-1)) = g(3) = 6 \cdot 3 - 5 = 18 - 5 = 13
%     \]
%     Using the formula: \( 12(-1) + 25 = -12 + 25 = 13 \). Correct.
%   \item For \( x = 0 \) (Case 2):
%     \[
%     f(0) = 0^3 + 6 = 6, \quad g(f(0)) = g(6) = 6 \cdot 6 - 5 = 36 - 5 = 31
%     \]
%     Using the formula: \( 6(0)^3 + 31 = 0 + 31 = 31 \). Correct.
%   \item For \( x = 1 \) (Case 2):
%     \[
%     f(1) = 1^3 + 6 = 1 + 6 = 7, \quad g(f(1)) = g(7) = 6 \cdot 7 - 5 = 42 - 5 = 37
%     \]
%     Using the formula: \( 6(1)^3 + 31 = 6 + 31 = 37 \). Correct.
% \end{itemize}
% The piecewise function correctly represents \( g \circ f \).

% \subsubsection*{Conclusion:}
% The composition is:
% \[
% (g \circ f)(x) = \begin{cases} 
% 12x + 25, & x < 0 \\
% 6x^3 + 31, & x \geq 0 
% \end{cases}
% \]
% \[
% \boxed{
% (g \circ f)(x) = \begin{cases} 
% 12x + 25, & x < 0 \\
% 6x^3 + 31, & x \geq 0 
% \end{cases}
% }
% \]
% % D1
\section{Formulate Corresponding Proof Principles to Prove Properties about Defined Sets (D1)}
\label{sec:D1}

% \subsubsection*{The Problem:}
% Show that if \( A \), \( B \), and \( C \) are sets, then:
% \[
% \overline{A \cup B \cup C} = \overline{A} \cap \overline{B} \cap \overline{C}
% \]
% In this section, we will prove the equality in two ways: by demonstrating that each side is a subset of the other and by verifying the equality using a membership table.

% \subsection{Demonstrate that each side is a subset of the other side}

% \subsubsection*{Given:}
% Let \( A \), \( B \), and \( C \) be subsets of a universal set \( U \). We need to prove:
% \[
% \overline{A \cup B \cup C} = \overline{A} \cap \overline{B} \cap \overline{C}
% \]
% where \( \overline{S} \) denotes the complement of set \( S \) with respect to \( U \), i.e., \( \overline{S} = \{ x \in U \mid x \notin S \} \).

% \subsubsection*{Proof Strategy:}
% To prove set equality, we show that each side is a subset of the other \cite{epp2020}:
% \begin{itemize}
%   \item Show \( \overline{A \cup B \cup C} \subseteq \overline{A} \cap \overline{B} \cap \overline{C} \): If \( x \in \overline{A \cup B \cup C} \), then \( x \in \overline{A} \cap \overline{B} \cap \overline{C} \).
%   \item Show \( \overline{A} \cap \overline{B} \cap \overline{C} \subseteq \overline{A \cup B \cup C} \): If \( x \in \overline{A} \cap \overline{B} \cap \overline{C} \), then \( x \in \overline{A \cup B \cup C} \).
% \end{itemize}
% This establishes \( \overline{A \cup B \cup C} = \overline{A} \cap \overline{B} \cap \overline{C} \).

% \subsubsection*{Proof:}
% \textbf{Part 1: Show \( \overline{A \cup B \cup C} \subseteq \overline{A} \cap \overline{B} \cap \overline{C} \).}
% Let \( x \in \overline{A \cup B \cup C} \). By definition of the complement:
% \[
% x \in \overline{A \cup B \cup C} \implies x \notin A \cup B \cup C
% \]
% By the definition of union, \( x \notin A \cup B \cup C \) means:
% \[
% x \notin A \text{ and } x \notin B \text{ and } x \notin C
% \]
% This implies:
% \[
% x \in \overline{A} \text{ and } x \in \overline{B} \text{ and } x \in \overline{C} \implies x \in \overline{A} \cap \overline{B} \cap \overline{C}
% \]
% Thus, \( \overline{A \cup B \cup C} \subseteq \overline{A} \cap \overline{B} \cap \overline{C} \).

% \textbf{Part 2: Show \( \overline{A} \cap \overline{B} \cap \overline{C} \subseteq \overline{A \cup B \cup C} \).}
% Let \( x \in \overline{A} \cap \overline{B} \cap \overline{C} \). By definition of intersection:
% \[
% x \in \overline{A} \cap \overline{B} \cap \overline{C} \implies x \in \overline{A} \text{ and } x \in \overline{B} \text{ and } x \in \overline{C}
% \]
% This means:
% \[
% x \notin A \text{ and } x \notin B \text{ and } x \notin C \implies x \notin A \cup B \cup C \implies x \in \overline{A \cup B \cup C}
% \]
% Thus, \( \overline{A} \cap \overline{B} \cap \overline{C} \subseteq \overline{A \cup B \cup C} \).

% Since both inclusions hold, we conclude:
% \[
% \overline{A \cup B \cup C} = \overline{A} \cap \overline{B} \cap \overline{C}
% \]

% \subsubsection*{Verification:}
% To verify, consider the logical equivalence of De Morgan’s Law. For any element \( x \):
% \begin{itemize}
%   \item \( x \in \overline{A \cup B \cup C} \) means \( x \notin A \cup B \cup C \), i.e., \( x \) is not in \( A \), \( B \), or \( C \).
%   \item \( x \in \overline{A} \cap \overline{B} \cap \overline{C} \) means \( x \in \overline{A} \), \( x \in \overline{B} \), and \( x \in \overline{C} \), i.e., \( x \notin A \), \( x \notin B \), and \( x \notin C \).
% \end{itemize}
% Both expressions describe the same set of elements, confirming the equality.

% \subsubsection*{Conclusion:}
% The set equality \( \overline{A \cup B \cup C} = \overline{A} \cap \overline{B} \cap \overline{C} \) is proven by showing mutual subset inclusion.

% \subsection{Verify the equality using a membership table}

% \subsubsection*{Given:}
% Let \( A \), \( B \), and \( C \) be subsets of a universal set \( U \). We need to verify:
% \[
% \overline{A \cup B \cup C} = \overline{A} \cap \overline{B} \cap \overline{C}
% \]
% using a membership table, which lists all possible combinations of membership for an element \( x \in U \) in sets \( A \), \( B \), and \( C \).

% \subsubsection*{Formula:}
% A membership table lists whether an element \( x \) belongs to each set (1 for in, 0 for not in) and evaluates both sides of the equality \cite{rosen2019}. For three sets, there are \( 2^3 = 8 \) combinations. The table compares membership in \( \overline{A \cup B \cup C} \) and \( \overline{A} \cap \overline{B} \cap \overline{C} \). If the columns match, the equality holds.

% \subsubsection*{Substitution:}
% Construct the membership table with columns for \( A \), \( B \), \( C \), \( A \cup B \cup C \), \( \overline{A \cup B \cup C} \), \( \overline{A} \), \( \overline{B} \), \( \overline{C} \), and \( \overline{A} \cap \overline{B} \cap \overline{C} \).

% \begin{center}
% \begin{tabular}{|c|c|c|c|c|c|c|c|c|}
% \hline
% \( A \) & \( B \) & \( C \) & \( A \cup B \cup C \) & \( \overline{A \cup B \cup C} \) & \( \overline{A} \) & \( \overline{B} \) & \( \overline{C} \) & \( \overline{A} \cap \overline{B} \cap \overline{C} \) \\
% \hline
% 1 & 1 & 1 & 1 & 0 & 0 & 0 & 0 & 0 \\
% 1 & 1 & 0 & 1 & 0 & 0 & 0 & 1 & 0 \\
% 1 & 0 & 1 & 1 & 0 & 0 & 1 & 0 & 0 \\
% 1 & 0 & 0 & 1 & 0 & 0 & 1 & 1 & 0 \\
% 0 & 1 & 1 & 1 & 0 & 1 & 0 & 0 & 0 \\
% 0 & 1 & 0 & 1 & 0 & 1 & 0 & 1 & 0 \\
% 0 & 0 & 1 & 1 & 0 & 1 & 1 & 0 & 0 \\
% 0 & 0 & 0 & 0 & 1 & 1 & 1 & 1 & 1 \\
% \hline
% \end{tabular}
% \end{center}

% \textbf{Explanation of columns:}
% \begin{itemize}
%   \item \( A \), \( B \), \( C \): 1 if \( x \in \) the set, 0 if \( x \notin \) the set.
%   \item \( A \cup B \cup C \): 1 if \( x \in A \), \( x \in B \), or \( x \in C \), else 0.
%   \item \( \overline{A \cup B \cup C} \): 1 if \( x \notin A \cup B \cup C \), else 0.
%   \item \( \overline{A} \), \( \overline{B} \), \( \overline{C} \): 1 if \( x \notin \) the set, else 0.
%   \item \( \overline{A} \cap \overline{B} \cap \overline{C} \): 1 if \( \overline{A} = 1 \), \( \overline{B} = 1 \), and \( \overline{C} = 1 \), else 0.
% \end{itemize}

% \subsubsection*{Verification:}
% The columns for \( \overline{A \cup B \cup C} \) and \( \overline{A} \cap \overline{B} \cap \overline{C} \) are identical (0, 0, 0, 0, 0, 0, 0, 1). This confirms that an element is in \( \overline{A \cup B \cup C} \) if and only if it is in \( \overline{A} \cap \overline{B} \cap \overline{C} \), verifying the equality.

% \subsubsection*{Conclusion:}
% The membership table shows that \( \overline{A \cup B \cup C} = \overline{A} \cap \overline{B} \cap \overline{C} \) for all possible membership combinations.

% \[
% \boxed{\overline{A \cup B \cup C} = \overline{A} \cap \overline{B} \cap \overline{C}}
% \]

% D1

\chapter{Analyze Mathematical Structures of Objects Using Graph Theory (LO2)}
\label{chap:LO2}

% P3
\section{Model Contextualized Problems Using Trees, both Quantitatively and Qualitatively (P3)}
\label{sec:P3}

\subsection{What is a Binary Tree?}

A binary tree is a hierarchical data structure in which each node has at most two children, referred to as the left child and the right child \cite{epp2020}. Formally, a binary tree is either empty or consists of a root node connected to a left subtree and a right subtree, both of which are binary trees. Binary trees are widely used in computer science for tasks such as searching, sorting, and hierarchical data representation due to their efficient structure for operations like traversal and insertion \cite{rosen2019}.

Key properties include:
\begin{itemize}
  \item \textbf{Nodes}: Each node contains a value and pointers to its left and right children (possibly null).
  \item \textbf{Height}: The length of the longest path from the root to a leaf.
  \item \textbf{Depth}: The length of the path from the root to a specific node.
  \item \textbf{Leaves}: Nodes with no children.
\end{itemize}
Binary trees are foundational in applications like binary search trees, expression trees, and heap data structures.

\subsection{Common Types of Binary Trees}

Several types of binary trees are commonly used, each with specific properties suited to particular applications \cite{rosen2019}:

\begin{itemize}
  \item \textbf{Complete Binary Tree}: A binary tree in which all levels, except possibly the last, are fully filled, and all nodes in the last level are as far left as possible. For a complete binary tree with height \( h \), the number of nodes \( n \) satisfies \( 2^h \leq n < 2^{h+1} \).
  \item \textbf{Full Binary Tree}: Every node has either zero or two children. The number of leaves is \( \frac{n+1}{2} \), where \( n \) is the total number of nodes.
  \item \textbf{Balanced Binary Tree}: The heights of the left and right subtrees of every node differ by at most one. This ensures \( O(\log n) \) time for operations like search and insertion (e.g., AVL trees, Red-Black trees).
  \item \textbf{Binary Search Tree (BST)}: A binary tree where the left subtree of a node contains values less than the node’s value, and the right subtree contains values greater, enabling efficient searching and sorting.
  \item \textbf{Heap}: A complete binary tree where each node’s value is greater than or equal to (max-heap) or less than or equal to (min-heap) its children’s values, used in priority queues and sorting (e.g., heapsort).
\end{itemize}

These types are tailored to specific computational needs, balancing efficiency and structural constraints.


\subsection{Discuss Two Notable Instances of Binary Trees}

We discuss two notable applications of binary trees: Binary Search Trees (BSTs) for database indexing and Min-Heaps for priority queue implementations. Each is analyzed quantitatively (e.g., height, node count) and qualitatively (context, benefits, limitations).

\subsubsection*{Instance 1: Binary Search Tree for Database Indexing}

\textbf{Quantitative Analysis:}
A Binary Search Tree (BST) is used to index a database with \( n \) records. Assume we have \( n = \overline{9b} = 96 \) records (using \( b = 6 \)) to index, with keys inserted in a random order to maintain approximate balance \citep{rosen2019}. For a balanced BST:
\begin{itemize}
  \item \textbf{Height}: The height \( h \) of a balanced BST with \( n = 96 \) nodes is approximately \( \lfloor \log_2(96) \rfloor \approx 6 \), since \( 2^6 = 64 \leq 96 < 128 = 2^7 \).
  \item \textbf{Search Time}: The average time complexity for search, insertion, and deletion is \( O(\log n) \approx O(\log 96) \approx O(6.58) \).
  \item \textbf{Node Count}: With \( a = 5 \), assume we query the top \( \overline{2a} = 25 \) keys. The number of comparisons per search is at most the height plus one, i.e., \( 6 + 1 = 7 \).
\end{itemize}
Thus, a BST with 96 nodes has a height of approximately 6, enabling efficient searches in about 7 comparisons per query.

\textbf{Qualitative Analysis:}
\begin{itemize}
  \item \textbf{Context}: BSTs are used in database systems to index records, allowing fast retrieval of data based on key values (e.g., user IDs in a customer database). For example, a retail database might use a BST to index customer transactions by order ID.
  \item \textbf{Benefits}: BSTs provide efficient search, insertion, and deletion operations when balanced, with \( O(\log n) \) complexity. They support dynamic updates, unlike static arrays, and are simple to implement.
  \item \textbf{Limitations}: If insertions are not randomized, the BST can become unbalanced (e.g., resembling a linked list), leading to \( O(n) \) worst-case complexity. Self-balancing BSTs (e.g., AVL, Red-Black trees) mitigate this but add complexity. Additionally, BSTs may not scale well for very large datasets compared to B-trees used in modern databases.
\end{itemize}

\subsubsection*{Instance 2: Min-Heap for Priority Queue in Task Scheduling}

\textbf{Quantitative Analysis:}
A Min-Heap is used to implement a priority queue for scheduling \( n = \overline{11b} = 116 \) tasks (using \( b = 6 \)) in a system, where lower values indicate higher priority \cite{rosen2019}. The heap is a complete binary tree:
\begin{itemize}
  \item \textbf{Height}: For \( n = 116 \), the height \( h = \lfloor \log_2(116) \rfloor \approx 6 \), since \( 2^6 = 64 \leq 116 < 128 = 2^7 \).
  \item \textbf{Operations}: Inserting a task or extracting the minimum takes \( O(\log n) \approx O(\log 116) \approx O(6.86) \). Assume \( a = 5 \), and we perform \( \overline{1a} = 15 \) extractions (e.g., processing the top 15 tasks). Total time is \( 15 \cdot O(\log 116) \approx 15 \cdot 6.86 \approx 103 \) operations.
  \item \textbf{Node Count}: The heap stores 116 tasks, with the minimum priority task at the root.
\end{itemize}
Thus, a Min-Heap with 116 tasks has a height of 6, allowing efficient task prioritization.

\textbf{Qualitative Analysis:}
\begin{itemize}
  \item \textbf{Context}: Min-Heaps are used in operating systems for task scheduling, where tasks with the highest priority (lowest value) are executed first. For example, a real-time system might use a Min-Heap to prioritize critical processes.
  \item \textbf{Benefits}: Min-Heaps ensure \( O(\log n) \) insertion and deletion of the minimum element, ideal for dynamic priority queues. Their complete binary tree structure minimizes memory usage and supports efficient array-based implementations.
  \item \textbf{Limitations}: Min-Heaps are less effective for searching arbitrary elements (requiring \( O(n) \) time). They are also sensitive to priority ties, which may require additional logic to resolve. For large-scale systems, more complex structures like Fibonacci heaps may offer better amortized performance.
\end{itemize}

% \subsubsection*{Conclusion:}

\section{Use Dijkstra’s Algorithm to Find a Shortest Path Spanning Tree in Graph (P4)}
\label{sec:P4}

% \subsection{State Dijkstra’s Algorithm in an undirected graph}
% \begin{figure}[h!]
%   \centering
%   \includegraphics[scale=1]{./images/P4graph.png}
%   \caption{The given undirected graph}
%   \label{fig:p4graph}
% \end{figure}
% \subsubsection*{Description:}
% Dijkstra’s algorithm finds the shortest path from a single source vertex to all other vertices in a weighted, undirected graph with non-negative edge weights \cite{rosen2019}. In an undirected graph, edges are bidirectional, meaning the weight from vertex \( u \) to \( v \) equals the weight from \( v \) to \( u \). The algorithm proceeds as follows:

% \begin{enumerate}
%   \item Initialize a distance array \( d \) where \( d[source] = 0 \) and \( d[v] = \infty \) for all other vertices \( v \).
%   \item Maintain a set of unvisited vertices \( S \).
%   \item While \( S \) is not empty:
%     \begin{enumerate}
%       \item Select the unvisited vertex \( u \) with the minimum \( d[u] \).
%       \item Mark \( u \) as visited (remove from \( S \)).
%       \item For each unvisited neighbor \( v \) of \( u \):
%         \begin{itemize}
%           \item Update \( d[v] = \min(d[v], d[u] + w(u, v)) \), where \( w(u, v) \) is the weight of edge \( (u, v) \).
%         \end{itemize}
%     \end{enumerate}
%   \item The value \( d[v] \) represents the shortest path length from the source to \( v \) for all \( v \).
% \end{enumerate}

% The algorithm constructs a shortest path spanning tree rooted at the source, with a time complexity of \( O((V + E) \log V) \) using a priority queue, where \( V \) is the number of vertices and \( E \) is the number of edges.

\subsection{Apply Dijkstra’s algorithm to determine the shortest path length between vertices \( A \) and \( Z \) in the provided weighted graph}

% \subsubsection*{Given:}
% Consider the undirected weighted graph provided, with vertices \( A, B, C, D, E, F, G, Z \) and the following edges with weights:
% \begin{itemize}
%   \item \( A-B = a \), \( A-C = 4 \)
%   \item \( B-D = 6 \)
%   \item \( C-D = 4 \), \( C-E = 7 \)
%   \item \( D-E = 1 \), \( D-F = 6 \)
%   \item \( E-F = b \), \( E-G = 6 \)
%   \item \( F-G = 3 \), \( F-Z = 8 \)
%   \item \( G-Z = 5 \)
% \end{itemize}
% Using \( a = 5 \) and \( b = 6 \) from the ID (BD00536), the weights are:
% \begin{itemize}
%   \item \( A-B = 5 \), \( A-C = 4 \)
%   \item \( B-D = 6 \)
%   \item \( C-D = 4 \), \( C-E = 7 \)
%   \item \( D-E = 1 \), \( D-F = 6 \)
%   \item \( E-F = 6 \), \( E-G = 6 \)
%   \item \( F-G = 3 \), \( F-Z = 8 \)
%   \item \( G-Z = 5 \)
% \end{itemize}
% The source vertex is \( A \), and the target is \( Z \). We apply Dijkstra’s algorithm to find the shortest path length from \( A \) to \( Z \).

% \subsubsection*{Steps:}
% Initialize:
% - \( d[A] = 0 \), \( d[B] = d[C] = d[D] = d[E] = d[F] = d[G] = d[Z] = \infty \)
% - Unvisited set \( S = \{A, B, C, D, E, F, G, Z\} \)

% Iterate:
% \begin{center}
% \begin{tabular}{|c|c|c|c|c|}
% \hline
% Step & Current Vertex (\( u \)) & \( d[u] \) & Unvisited Set \( S \) and Distances & Updated \( d \) Values \\
% \hline
% 1 & \( A \) & 0 & \( d[B] = 5 \), \( d[C] = 4 \) & \( d[B] = 5 \), \( d[C] = 4 \) \\
% 2 & \( C \) & 4 & \( d[B] = 5 \), \( d[D] = 8 \), \( d[E] = 11 \) & \( d[D] = 8 \), \( d[E] = 11 \) \\
% 3 & \( B \) & 5 & \( d[D] = 8 \), \( d[E] = 11 \) & \( d[D] = 8 \) (no change) \\
% 4 & \( D \) & 8 & \( d[E] = 9 \), \( d[F] = 14 \) & \( d[E] = 9 \), \( d[F] = 14 \) \\
% 5 & \( E \) & 9 & \( d[F] = 14 \), \( d[G] = 15 \) & \( d[F] = 14 \), \( d[G] = 15 \) \\
% 6 & \( F \) & 14 & \( d[G] = 15 \), \( d[Z] = 22 \) & \( d[G] = 15 \), \( d[Z] = 22 \) \\
% 7 & \( G \) & 15 & \( d[Z] = 20 \) & \( d[Z] = 20 \) \\
% 8 & \( Z \) & 20 & \( \emptyset \) & - \\
% \hline
% \end{tabular}
% \end{center}

% \begin{itemize}
%   \item \textbf{Step 1}: Select \( A \) (\( d[A] = 0 \)). Update \( d[B] = 5 \) (via \( A-B \)), \( d[C] = 4 \) (via \( A-C \)). \( S = \{B, C, D, E, F, G, Z\} \).
%   \item \textbf{Step 2}: Select \( C \) (\( d[C] = 4 \)). Update \( d[D] = 4 + 4 = 8 \) (via \( C-D \)), \( d[E] = 4 + 7 = 11 \) (via \( C-E \)). \( S = \{B, D, E, F, G, Z\} \).
%   \item \textbf{Step 3}: Select \( B \) (\( d[B] = 5 \)). Update \( d[D] = \min(8, 5 + 6) = 8 \) (via \( B-D \)). \( S = \{D, E, F, G, Z\} \).
%   \item \textbf{Step 4}: Select \( D \) (\( d[D] = 8 \)). Update \( d[E] = \min(11, 8 + 1) = 9 \) (via \( D-E \)), \( d[F] = \min(\infty, 8 + 6) = 14 \) (via \( D-F \)). \( S = \{E, F, G, Z\} \).
%   \item \textbf{Step 5}: Select \( E \) (\( d[E] = 9 \)). Update \( d[F] = \min(14, 9 + 6) = 14 \) (via \( E-F \)), \( d[G] = \min(\infty, 9 + 6) = 15 \) (via \( E-G \)). \( S = \{F, G, Z\} \).
%   \item \textbf{Step 6}: Select \( F \) (\( d[F] = 14 \)). Update \( d[G] = \min(15, 14 + 3) = 15 \) (via \( F-G \)), \( d[Z] = \min(\infty, 14 + 8) = 22 \) (via \( F-Z \)). \( S = \{G, Z\} \).
%   \item \textbf{Step 7}: Select \( G \) (\( d[G] = 15 \)). Update \( d[Z] = \min(22, 15 + 5) = 20 \) (via \( G-Z \)). \( S = \{Z\} \).
%   \item \textbf{Step 8}: Select \( Z \) (\( d[Z] = 20 \)). No unvisited neighbors. \( S = \emptyset \).
% \end{itemize}

% Final distances: \( d[A] = 0 \), \( d[B] = 5 \), \( d[C] = 4 \), \( d[D] = 8 \), \( d[E] = 9 \), \( d[F] = 14 \), \( d[G] = 15 \), \( d[Z] = 20 \).

% \subsubsection*{Verification:}
% Check the shortest path:
% \begin{itemize}
%   \item Path \( A \to C \to D \to E \to G \to Z \): \( 4 + 4 + 1 + 6 + 5 = 20 \), matches \( d[Z] = 20 \).
%   \item Path \( A \to C \to D \to E \to F \to Z \): \( 4 + 4 + 1 + 6 + 8 = 23 \).
%   \item Path \( A \to B \to D \to E \to G \to Z \): \( 5 + 6 + 1 + 6 + 5 = 23 \).
%   \item Path \( A \to C \to E \to G \to Z \): \( 4 + 7 + 6 + 5 = 22 \).
% \end{itemize}

% The minimum is 20, confirming the algorithm’s result via \( A \to C \to D \to E \to G \to Z \).

% \subsubsection*{Conclusion:}
% The shortest path length from \( A \) to \( Z \) is 20, achieved via the path \( A \to C \to D \to E \to G \to Z \) with weights 4, 4, 1, 6, and 5.

% \[
% \boxed{20}
% \]



\section{Assess whether an Eulerian and Hamiltonian Circuit Exists in an Undirected Graph (M2)}
\label{sec:M2}



\section{Construct a Proof of the Five-Color Theorem (D2)}
\label{sec:D2}

% \subsubsection*{The Problem:}
% Prove the Five-Color Theorem: Every planar graph can be colored with at most five colors such that no two adjacent vertices share the same color.

% \subsubsection*{Given:}
% A planar graph \( G = (V, E) \) is a graph that can be embedded in the plane without edge crossings. A proper coloring assigns a color to each vertex such that no two vertices connected by an edge have the same color. We aim to show that \( G \) is 5-colorable, meaning it can be colored using at most five colors (e.g., \{1, 2, 3, 4, 5\}) \cite{rosen2019}.

% \subsubsection*{Proof Strategy:}
% We use mathematical induction on the number of vertices \( |V| = n \) in the planar graph \( G \), leveraging the fact that every planar graph has at least one vertex of degree at most 5 \cite{diestel2017}. The strategy is:
% \begin{itemize}
%   \item \textbf{Base Case}: Verify the theorem for small graphs (e.g., \( n \leq 5 \)).
%   \item \textbf{Inductive Step}: Assume the theorem holds for graphs with fewer than \( n \) vertices. For a graph with \( n \) vertices, remove a vertex of degree at most 5, color the resulting graph with at most five colors, and reintroduce the vertex. For vertices of degree 5, use a Kempe chain argument to adjust colors if necessary to ensure a valid coloring \cite{rosen2019}.
% \end{itemize}
% This approach ensures a proper 5-coloring \cite{epp2020}.

% \subsubsection*{Proof:}
% We proceed by induction on the number of vertices \( n = |V| \).

% \textbf{Base Case (\( n \leq 5 \)):}
% For a planar graph with \( n \leq 5 \) vertices, the number of edges is limited by planarity. By Euler’s formula for planar graphs (\( |V| - |E| + |F| = 2 \), where \( |F| \) is the number of faces), and since each face (including the outer face) has at least three edges, we have \( 3|F| \leq 2|E| \). Combining with Euler’s formula, for a simple planar graph:
% \[
% |E| \leq 3n - 6 \quad \text{for } n \geq 3
% \]
% For \( n \leq 5 \), the maximum degree is small, and the graph can be colored with at most five colors. For example:
% \begin{itemize}
%   \item If \( n = 1 \): A single vertex requires one color.
%   \item If \( n = 2 \): Two vertices (with at most one edge) need at most two colors.
%   \item If \( n = 3 \): A triangle (complete graph \( K_3 \)) needs three colors.
%   \item If \( n = 4 \): A planar graph like \( K_4 \) (tetrahedron) needs at most four colors.
%   \item If \( n = 5 \): A planar graph (e.g., a cycle \( C_5 \)) needs at most three colors.
% \end{itemize}
% Since \( n \leq 5 \), five colors are sufficient for the base case.

% \textbf{Inductive Hypothesis:}
% Assume that every planar graph with \( k < n \) vertices (\( n \geq 6 \)) can be properly colored with at most five colors.

% \textbf{Inductive Step:}
% Consider a planar graph \( G = (V, E) \) with \( n \) vertices. By a standard result in graph theory, every planar graph has at least one vertex of degree at most 5 \cite{diestel2017}. Choose a vertex \( v \in V \) with degree \( \deg(v) \leq 5 \), and let its neighbors be \( \{ u_1, u_2, \ldots, u_k \} \), where \( k \leq 5 \).

% Form the graph \( G' = G - v \), obtained by removing \( v \) and its incident edges. Since \( G' \) is planar and has \( n-1 \) vertices, by the inductive hypothesis, \( G' \) can be properly colored with at most five colors, say \{1, 2, 3, 4, 5\}.

% Now, reintroduce vertex \( v \) to form \( G \). We need to assign a color to \( v \) such that it differs from the colors of its neighbors \( u_1, \ldots, u_k \).

% \begin{itemize}
%   \item \textbf{Case 1: \( \deg(v) \leq 4 \)}:
%     The neighbors \( u_1, \ldots, u_k \) (where \( k \leq 4 \)) use at most four colors. Since there are five colors available, at least one color (say, color 1) is not used by any neighbor. Assign \( c(v) = 1 \). This ensures a proper coloring, as \( v \)’s color differs from its neighbors’ colors, and the coloring of \( G' \) remains valid.

%   \item \textbf{Case 2: \( \deg(v) = 5 \)}:
%     Suppose the neighbors \( u_1, u_2, u_3, u_4, u_5 \) are colored with \( c(u_i) = i \) for \( i = 1, 2, 3, 4, 5 \), using all five colors. If the neighbors use fewer than five distinct colors (e.g., two neighbors share a color), then at least one color is available for \( v \), and we can assign it directly. Assume the worst case: all five neighbors have distinct colors.

%     To resolve this, use a Kempe chain argument \cite{rosen2019}:
%     \begin{itemize}
%       \item Consider the subgraph \( G'_{1,3} \) of \( G' \) induced by vertices colored 1 or 3 (i.e., vertices with colors 1 or 3 and edges between them where one endpoint is color 1 and the other is color 3).
%       \item Let \( u_1 \) have color 1 and \( u_3 \) have color 3. Analyze the connected component in \( G'_{1,3} \):
%         \begin{itemize}
%           \item \textbf{Subcase A: \( u_1 \) and \( u_3 \) are in different connected components of \( G'_{1,3} \).}
%             Swap the colors 1 and 3 in the component containing \( u_1 \). Since \( G'_{1,3} \) is bipartite (its vertices are partitioned into colors 1 and 3, with edges only between different colors), swapping colors 1 and 3 in \( u_1 \)’s component preserves the proper coloring of \( G' \). Now, \( u_1 \) has color 3, and since \( u_3 \) is in a different component, its color remains 3. Thus, color 1 is no longer used by any neighbor of \( v \). Assign \( c(v) = 1 \).

%           \item \textbf{Subcase B: \( u_1 \) and \( u_3 \) are in the same connected component of \( G'_{1,3} \).}
%             There exists a path \( P \) in \( G'_{1,3} \) from \( u_1 \) to \( u_3 \) with alternating colors 1 and 3. In the planar embedding of \( G \), the edges \( (v, u_1) \) and \( (v, u_3) \), combined with path \( P \), form a cycle \( C \) (since \( v \) connects to both \( u_1 \) and \( u_3 \)). By planarity, this cycle separates the plane into an interior and exterior region.

%             Now, consider neighbors \( u_2 \) (color 2) and \( u_4 \) (color 4). Since \( C \) is a cycle in the planar embedding, \( u_2 \) and \( u_4 \) must lie in different regions (e.g., one inside and one outside \( C \)). Thus, in the subgraph \( G'_{2,4} \) induced by vertices colored 2 or 4, there cannot be a path from \( u_2 \) to \( u_4 \), as such a path would cross \( C \), violating planarity (since \( C \)’s vertices are colored 1 and 3, not 2 or 4).

%             Therefore, \( u_2 \) and \( u_4 \) are in different connected components of \( G'_{2,4} \). Swap colors 2 and 4 in the component containing \( u_2 \). This preserves the proper coloring of \( G' \), as \( G'_{2,4} \) is bipartite. Now, \( u_2 \) has color 4, and \( u_4 \)’s color remains 4 (since it’s in a different component). Thus, color 2 is no longer used by any neighbor of \( v \). Assign \( c(v) = 2 \).
%         \end{itemize}
%     \end{itemize}
%     In both subcases, we free up a color for \( v \), ensuring a proper 5-coloring of \( G \).
% \end{itemize}

% Thus, in all cases (\( \deg(v) \leq 4 \) or \( \deg(v) = 5 \)), we can assign a color to \( v \) such that the coloring of \( G \) is proper.

% By the principle of mathematical induction \cite{epp2020}, the Five-Color Theorem holds for all planar graphs.

% \subsubsection*{Verification:}
% The key property is that every planar graph has a vertex of degree at most 5, derived from \( |E| \leq 3n - 6 \) and the degree sum \( 2|E| \leq 6n - 12 \), so the average degree is less than 6. The induction step ensures that:
% \begin{itemize}
%   \item For \( \deg(v) \leq 4 \), at least one of the five colors is available.
%   \item For \( \deg(v) = 5 \), the Kempe chain argument guarantees that we can recolor \( G' \) to free up a color for \( v \), leveraging planarity to ensure that neighbors like \( u_2 \) and \( u_4 \) are in different components when \( u_1 \) and \( u_3 \) are connected.
% \end{itemize}
% Testing on small planar graphs (e.g., \( K_4 \), \( C_5 \)) confirms that fewer than five colors often suffice, but the proof guarantees five colors are always sufficient. The Kempe chain argument addresses the critical case where all five neighbors have distinct colors, ensuring the proof is complete.

% \subsubsection*{Conclusion:}
% The Five-Color Theorem is proven: every planar graph can be properly colored with at most five colors.
% \[
% \boxed{\text{Every planar graph is 5-colorable}}
% \]


\chapter{Investigate Solutions to Problem Situations Using the Application of Boolean Algebra (LO3)}
\label{chap:LO3}

\section{Diagram a Binary Problem in the Application of Boolean Algebra (P5)}
\label{sec:P5}

% \subsection{Introduction to Boolean Algebra in Binary Problems}

% Boolean algebra is a mathematical framework used to analyze and design systems involving binary variables, which take values of 0 (false) or 1 (true)\cite{rosen2019}. It is fundamental in computer science for modeling and solving binary problems, particularly in digital circuit design and logic optimization. This section diagrams binary problems in two real-world domains—digital circuit design for a voting system and network security for access control—illustrating number representation, logic gate usage, and their practical applications.

% \subsection{Domain 1: Digital Circuit Design for a Voting System}
% \subsubsection*{Problem Description}

% Consider a voting system for a three-member committee where a decision is approved if at least two members vote in favor. Each member’s vote is a binary input: 1 (approve) or 0 (reject). The system outputs 1 if the decision is approved, and 0 otherwise. Let the votes of members A, B, and C be represented by Boolean variables \( A \), \( B \), and \( C \), respectively. The problem is to design a circuit that computes the majority vote using Boolean algebra.

% \subsubsection*{Number Representation}

% The inputs \( A \), \( B \), and \( C \) are binary (0 or 1), representing the votes. The output \( Y \) is also binary, where:
% - \( Y = 1 \): At least two of \( A \), \( B \), or \( C \) are 1 (decision approved).
% - \( Y = 0 \): Fewer than two are 1 (decision rejected).

% The Boolean function for the majority vote can be expressed as:
% \[
% Y = (A \land B) \lor (A \land C) \lor (B \land C)
% \]
% This equation outputs 1 when at least two inputs are 1, capturing combinations \( A \land B \), \( A \land C \), or \( B \land C \).

% \subsubsection*{Logic Gate Diagram}

% The circuit is implemented using AND, OR, and NOT gates:
% - \textbf{AND gates}: Compute pairwise conjunctions \( A \land B \), \( A \land C \), and \( B \land C \).
% - \textbf{OR gate}: Combines the outputs of the AND gates to produce \( Y \).

% The logic gate diagram is as follows:

% \begin{figure}[h!]
%   \centering
%   \includegraphics[scale=0.8]{./images/P4graph.png}
%   \caption{Logic gate diagram for the majority voting system}
%   \label{fig:voting_circuit}
% \end{figure}

% (Note: In practice, the diagram would be drawn using a tool like Logisim or LaTeX’s circuitikz package, showing three AND gates feeding into a single OR gate.)

% \subsubsection*{Practical Applications}

% This circuit has applications in:
% - \textbf{Fault-tolerant systems}: Used in redundant systems (e.g., aerospace control systems) to ensure decisions are made based on majority agreement among multiple processors.
% - \textbf{Democratic voting systems}: Implements electronic voting mechanisms in small committees or distributed systems.
% - \textbf{Error detection}: Similar logic is used in majority voting for error correction in data transmission, ensuring reliability \cite{rosen2019}.

% \subsubsection*{Qualitative Analysis}

% \begin{itemize}
%   \item \textbf{Benefits}: The circuit is simple, using only AND and OR gates, and is scalable for more inputs by extending the logic (e.g., for \( n \) voters, check combinations of at least \( \lceil n/2 \rceil \) votes). It provides fast, real-time decision-making.
%   \item \textbf{Limitations}: The circuit assumes binary inputs and does not handle weighted votes or tie-breaking mechanisms. For large systems, the number of AND gates grows combinatorially (\( \binom{n}{k} \)).
% \end{itemize}

% \subsection{Domain 2: Network Security for Access Control}

% \subsubsection*{Problem Description}

% In a secure facility, access is granted if the access card is swiped (\( C = 1 \)) OR the correct PIN is entered (\( P = 1 \)) AND the security system is NOT in maintenance mode (\( M = 0 \)). The system outputs 1 (unlock) or 0 (lock). The Boolean expression is:
% \[
% Y = C \lor (P \land \neg M)
% \]
% where \( C \), \( P \), and \( M \) are binary variables, and \( \neg M \) is the negation of \( M \).

% \subsubsection*{Number Representation}

% The variables are:
% \begin{itemize}
%   \item \( C \): 1 (card swiped), 0 (not swiped).
%   \item \( P \): 1 (correct PIN), 0 (incorrect PIN).
%   \item \( M \): 1 (maintenance mode), 0 (normal mode).
%   \item \( Y \): 1 (door unlocks), 0 (door remains locked).
% \end{itemize}

% The expression \( Y = C \lor (P \land \neg M) \) outputs 1 if either the card is swiped or both the PIN is correct and the system is not in maintenance mode.

% \subsubsection*{Logic Gate Diagram}

% The circuit uses:
% \begin{itemize}
%   \item \textbf{NOT gate}: Inverts \( M \) to produce \( \neg M \).
%   \item \textbf{AND gate}: Computes \( P \land \neg M \).
%   \item \textbf{OR gate}: Combines \( C \) with \( P \land \neg M \) to produce \( Y \).
% \end{itemize}

% \begin{figure}[h!]
%   \centering
%   \includegraphics[scale=0.8]{./images/P4graph.png}
%   \caption{Logic gate diagram for the access control system}
%   \label{fig:access_control_circuit}
% \end{figure}

% (Note: The diagram would show a NOT gate on \( M \), an AND gate for \( P \land \neg M \), and an OR gate combining \( C \) with the AND output.)

% \subsubsection*{Practical Applications}

% This circuit is used in:
% \begin{itemize}
%   \item \textbf{Security systems}: Controls access to buildings, data centers, or restricted areas, ensuring multiple authentication methods.
%   \item \textbf{Computer login systems}: Similar logic governs multi-factor authentication, combining passwords, biometrics, or tokens.
%   \item \textbf{Automated control systems}: Used in industrial settings to enforce safety protocols based on multiple conditions \cite{epp2020}.
% \end{itemize}

% \subsubsection*{Qualitative Analysis}

% \begin{itemize}
%   \item \textbf{Benefits}: The circuit is robust, combining multiple authentication methods to enhance security. It is simple to implement with basic gates and easily extensible (e.g., adding biometric inputs).
%   \item \textbf{Limitations}: The system assumes binary conditions and does not account for partial failures (e.g., card reader malfunctions). Additional logic may be needed for timeouts or emergency overrides.
% \end{itemize}

% \subsection{Conclusion}

% Boolean algebra effectively models binary problems in digital circuit design and network security. The voting system demonstrates how majority logic ensures reliable decision-making, while the access control system illustrates multi-factor authentication. Both use AND, OR, and NOT gates to implement Boolean functions, showcasing their versatility in real-world applications. These examples highlight the power of Boolean algebra in simplifying complex binary decisions \cite{rosen2019}.


\section{Produce a Truth Table and Its Corresponding Boolean Equation from an Applicable Scenario (P6)}
\label{sec:P6}

\section{Simplify a Boolean Equation Using Algebraic Methods (M3)}
\label{sec:M3}

\section{Design a Complex System Using Logic Gates (D1)}
\label{sec:D1}

\chapter{Explore Applicable Concepts within Abstract Algebra (LO4)}
\label{chap:LO4}

\section{Describe the Distinguishing Characteristics of Different Binary Operations that are Performed on the same Set (P7)}
\label{sec:P7}

\section{Determine the Order of a Group and the Order of a Subgroup in Given Examples (P8)}
\label{sec:P8}

\section{Validate whether a Given Set with a Binary Operation is indeed a Group (M4)}
\label{sec:M4}

\section{Explore, with the Aid of a Prepared Presentation, the Application of Group Theory Relevant to your Given Example (D4)}
\label{sec:D4}



\newpage
\section*{Conclusion}
\newpage
\section*{Evaluation}
\bibliography{references}

\end{document}

% Run as following order:
% pdflatex main 
% bibtex main
% pdflatex main
% pdflatex main


