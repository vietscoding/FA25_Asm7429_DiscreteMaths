\documentclass[12pt, a4paper, twoside]{report} % hoặc book
% Languages & font
\usepackage[english]{babel}

% Layout & page
\usepackage{geometry}
\usepackage{fancyhdr}

% Graphic & tables
\usepackage{graphicx}

% References & Citiations
\usepackage{natbib}

% Tiện ích lặt vặt
\usepackage{lipsum} % Văn bản vô nghĩa

\usepackage{blindtext}
\usepackage{titlesec}
\usepackage{setspace}
\usepackage{amsmath}

\geometry{
    top=1cm,          % Lề trên 1cm, bao gồm header
    bottom=1cm,       % Lề dưới 1cm, bao gồm footer
    left=1.25cm,
    right=1cm,
    headheight=12pt,  % Chiều cao header
    headsep=5mm,      % Khoảng cách từ header đến text
    footskip=8mm,     % Khoảng cách từ text đến footer
    includehead,      % Bao gồm header trong printable area
    includefoot       % Bao gồm footer trong printable area
}
% \usepackage{showframe} % Hiển thị khung lề
\usepackage{fancyhdr} % tùy chỉnh header & footer
\usepackage{layouts}   % Thêm gói layouts để hiển thị kích thước
% \usepackage{fontspec}
% \setmainfont{Calibri}

% References & Citiations
\usepackage{natbib}

\usepackage[hidelinks]{hyperref}

\onehalfspacing   % giãn dòng 1.5

\bibliographystyle{agsm}     % định dạng Harvard (có thể thay đổi)

% Đặt kiểu header/footer
\pagestyle{fancy}
\fancyhf{} % xóa sạch mặc định

% Header: odd/even (tự động lấy Chapter/Section)
\fancyhead[LO]{\leftmark}   % trang lẻ: Chapter
\fancyhead[RE]{\rightmark}  % trang chẵn: Section

% Footer: tác giả + số trang
\fancyfoot[L]{Written by: Doan Quoc Viet - BD00536} 
\fancyfoot[R]{\thepage}

% Làm Chapter và Section hiện đúng trong header
\renewcommand{\chaptermark}[1]{%
  \markboth{Chapter \thechapter: #1}{}} % Đặt \rightmark rỗng khi bắt đầu chương mới
\renewcommand{\sectionmark}[1]{%
  \markright{\thesection. #1}} % Cập nhật \rightmark khi có Section mới



\begin{document}
% Dùng title case cho các chapter và tiêu đề sau chapter (1.1)
% Không viết hoa: mạo từ (a, an, the), giới từ (in, of,...), liên từ đồng đẳng (and, or, but)
% Không chuyển động từ của tiêu đề sang danh động từ, trừ khi brief yêu cầu

% Hiển thị sơ đồ bố cục chi tiết (chỉ chạy một lần để xem)
%\currentpage
%\pagedesign

\tableofcontents 

\newpage


\section*{Introduction}
Mathematics is the foundation of the fields of software engineering and information technology (IT), where discrete mathematics plays an important role in solving complex computational problems. Discrete mathematics, which includes fields such as set theory, graph theory, Boolean algebra, and abstract algebra, equips software engineers with the tools to design efficient algorithms, model data structures, and ensure robust system performance. For example, concepts from discrete mathematics are integral to database query optimization, cryptographic protocols, network routing algorithms, and logic circuit design, enabling practical solutions to real-world challenges (Rosen, 2018).

The application of discrete mathematics in software engineering offers significant benefits. It allows developers to create optimized algorithms with improved time and space complexity, as seen in sorting and searching techniques. Furthermore, discrete structures such as graphs facilitate the modeling of relationships in social networks or communication systems. At the same time, Boolean algebra supports the development of reliable digital circuits, and abstract algebra helps understand symmetries in computational problems (Epp, 2010). By mastering these concepts, engineers can solve complex problems of accuracy and scalability. This assignment explores the practical applications of discrete mathematics in software engineering through a structured analysis organized into four chapters.

% Tham chiếu chéo

\begin{itemize}
  \item Chapter 1 focuses on set theory and its relevance to software engineering \hyperref[chap:LO1]{(LO1)}.
  \begin{itemize}
    \item In Section 1.1, operations on algebraic sets will be applied to solve a problem \hyperref[sec:P1]{(P1)}.
    \item Section 1.2 determines the cardinality of a multiset by factoring it, expressing the result as individual multisets \hyperref[sec:P1]{(P2)}.
    \item Section 1.3 discusses finding the inverse of a function using appropriate mathematical techniques \hyperref[sec:M1]{(M1)}.
    \item Finally, Section 1.4 develops proof principles for verifying properties of sets using membership tables and subset relations \hyperref[sec:D1]{(D1)}.
  \end{itemize}
  \item Chapter 2 reviews graph theory and its role in modeling mathematical structures \hyperref[chap:LO2]{(LO2)}.
  \begin{itemize}
    \item Section 2.1 models contextual problems with binary trees, analyzing two cases with quantitative and qualitative insights \hyperref[sec:P3]{(P3)}.
    \item Section 2.2 applies Dijkstra's algorithm to compute the length of the shortest path between two vertices in an undirected graph \hyperref[sec:P4]{(P4)}.
    \item Section 2.3 evaluates the existence of Euler and Hamiltonian cycles in an undirected graph \hyperref[sec:M2]{(M2)}.
    \item Finally, Section 2.4 constructs a proof of the Five Color Theorem \hyperref[sec:D4]{(D4)}.
  \end{itemize}
  \item Chapter 3 investigates solutions to problem situations using Boolean algebra (LO3).
  \begin{itemize}
    \item Section 3.1 discusses the use of Boolean algebra to solve binary problems in two diverse real-world domains, including real-world applications (P5).
    \item Section 3.2 generates truth tables and corresponding Boolean equations from application situations (P6).
    \item Section 3.3 simplifies Boolean equations using algebraic methods (M3).
    \item Finally, Section 3.4 designs a complex system using logic gates, such as constructing a circuit to detect divisibility by 3 in binary-coded decimal (D3).
  \end{itemize}
  \item Chapter 4 explores concepts applied in abstract algebra (LO4).
  \begin{itemize}
    \item Section 4.1 describes the distinguishing features of different binary operations performed on the same set (P7).
    \item Section 4.2 determines the order of a group and the order of a subgroup in given examples, including an application of Lagrange's theorem (P8).
    \item Section 4.3 confirms whether a given set with a binary operation forms a group (M4).
    \item Finally, Section 4.4 explores the application of group theory in computer science through a prepared presentation on a related topic, such as cryptography or algorithm design (D4).
  \end{itemize}

\end{itemize}



% All sources referenced in this assignment are documented in the References section.

\chapter{Examine Set Theory and Functions Applicable to Software Engineering (LO1)}

\label{chap:LO1}  % Đặt nhãn để tham chiếu chéo (cross-references)

\section{Perform Algebraic Set Operations in a Formulated Mathematical Problem (P1)}
% \sectionmark{Perform Algebraic Set Operations in a Formulated Mathematical Problem (P1)} 
\label{sec:P1}
In this assignment
\begin{itemize}
  \item the symbol $\overline{9b}$ denotes a two-digit natural number,
  \item the symbol $\overline{12a}$ represents a three-digit natural number.
\end{itemize}

For example, if $a=3$, then $\overline{12a} = 123$.

For each student, replace a and b with specific numerical values to obtain the final result.

According to the requirement above, I need to address problems in this assignment using my ID, which is BD00536. $a$ and $b$ ($a < b$) represents the two largest numbers in my ID, so $a$ = 5 and $b$ = 6.

I will address the issues listed below in the respective subsections in this Section:

\begin{itemize}
  \item Let $A$ and $B$ be two non-empty finite sets. Assume that cardinalities of the sets $A$, $B$, and $A \cap B$ are $\overline{9b}$, $\overline{2a}$, and $a + b$, respectively. Determine the cardinality of set $A \cup B$.
  \item Suppose $|A-B|=\overline{3a}$, $|A \cup B|=\overline {11b}$, and $|A \cap B|=\overline{1a}$. Determine $|B|$.
  \item At a local market, there are $\overline {35b}$ customers. Suppose $\overline {11a}$ have purchased fruits, $\overline {9b}$ have purchased vegetables, $\overline {8a}$ have purchased bakery items, $\overline{4b}$ have purchased both fruits and vegetables, $\overline{3b}$ have purchased both vegetables and bakery items, $\overline{2a}$ have purchased both fruits and bakery items, and $\overline{1a}$ have purchased all three categories. How many customers have not purchased anything?
\end{itemize}

In the following subsections I will address these issues and along the way use a = 5 and b = 6 as the two largest digits in my ID to demonstrate.


\subsection{Determine the cardinality of set $A \cup B$}
\subsubsection*{The Problem:}

  Let $A$ and $B$ be two non-empty finite sets. Assume that cardinalities of the sets $A$, $B$, and $A \cap B$ are $\overline{9b}$, $\overline{2a}$, and $a + b$, respectively. Determine the cardinality of set $A \cup B$.

  To solve this problem, I need to determine the cardinality of the set $A \cup B$, based on the given information about the cardinalities of the sets $A$, $B$, and $A \cap B$.

\subsubsection*{Given:}
  Using \(a = 5\) and \(b = 6\) from my ID (BD00536), the cardinalities are computed as follows:
  \begin{itemize}
    \item The cardinality of set \(A\) is \(\overline{9b} = 96\).
    \item The cardinality of set \(B\) is \(\overline{2a} = 25\).
    \item The cardinality of set \(A \cap B\) is \(a + b = 5 + 6 = 11\).
  \end{itemize}
  I need to find the cardinality of the set \(A \cup B\).

\subsubsection*{Formula:}
  The cardinality of the union of two sets is given by the inclusion-exclusion principle \citep{rosen2019}:
  \[
  |A \cup B| = |A| + |B| - |A \cap B|
  \]

\subsubsection*{Substitution:}
  Substituting the given values into the formula:
  \[
  |A \cup B| = 96 + 25 - 11
  \]

  Calculating step-by-step:
  \begin{enumerate}
    \item 96 + 25 = 121
    \item 121 - 11 = 110
  \end{enumerate}


  Thus, the cardinality of the set \( A \cup B \) is:
  \[
  |A \cup B| = 110
  \]

\subsubsection*{Verification:}
  To ensure the result is correct, I check the conditions:

\begin{itemize}
  \item \( |A \cap B| = 11 \), meaning the intersection \( A \cap B \) has 11 elements.
  \item Since \( |A \cap B| \leq |A| \) and \( |A \cap B| \leq |B| \), we have \( 11 \leq 96 \) and \( 11 \leq 25 \), both of which are satisfied.
  \item The sets \( A \) and \( B \) are non-empty and finite, which aligns with the assumptions.
\end{itemize}


Therefore, there are no contradictions, and the calculation is valid.

\subsubsection*{Conclusion:}
The cardinality of the set \( A \cup B \) is $\boxed{110}$.


\subsection{Determine $| B |$}

\subsubsection*{The Problem:}
  Suppose $|A-B|=\overline{3a}$, $|A \cup B|=\overline {11b}$, and $|A \cap B|=\overline{1a}$. Determine $|B|$.

\subsubsection*{Given:}
  Using \(a = 5\) and \(b = 6\) from my ID (BD00536), and based on the given problem we have the following data:
  \begin{itemize}
    \item $\vert A - B \vert = \overline{3a} = 35$
    \item $\vert A \cup B \vert = \overline{11b} = 116$
    \item $\vert A \cap B \vert = \overline{1a} = 15$
  \end{itemize}
  Let's determine $|B|$

\subsubsection*{Formula:}
  I will use the \textbf{principle of inclusion-exclusion} to determine $|B|$:
    \[
    |A \cup B| = |A| + |B| - |A \cap B| \tag{1}
    \]

\subsubsection*{Substitution:}
  \begin{align*}
    116 &= |A| + |B| - 15 //
    |A| + |B| &= 131. \tag{2}
  \end{align*}

  For \( |A - B| \):
  \[
  |A - B| = |A| - |A \cap B|.
  \]

  \[
  35 = |A| - 15.
  \]

  \[
  |A| = 35 + 15 = 50.
  \]

  Substitute \( |A| = 50 \) into equation (2):
  \[
  50 + |B| = 131.
  \]

  \[
  |B| = 131 - 50 = 81.
  \]

\subsubsection*{Verification:}
  \begin{itemize}
    \item $|A|=50, |B|=81,|A\cap B|=15.$
    \item Check: $|A \cup B| = |A| + |B| - |A \cap B| = 50 + 81 - 15 = 116$, which matches.
    \item Check: $|A - B| = |A| - |A \cap B| = 50 - 15 = 35$, which matches.
  \end{itemize}

\subsubsection*{Conclusion:}
Final answer: \boxed{81}

\subsection{How many customers have not purchased anything?}

\subsubsection*{The Problem:}
At a local market, there are $\overline {35b}$ customers. Suppose $\overline {11a}$ have purchased fruits, $\overline {9b}$ have purchased vegetables, $\overline {8a}$ have purchased bakery items, $\overline{4b}$ have purchased both fruits and vegetables, $\overline{3b}$ have purchased both vegetables and bakery items, $\overline{2a}$ have purchased both fruits and bakery items, and $\overline{1a}$ have purchased all three categories. How many customers have not purchased anything?
\subsubsection*{Given:}
  Using \(a = 5\) and \(b = 6\) from my ID (BD00536), and based on the given problem we have the following data:

  \begin{itemize}
    \item There are $\overline{35b} = 356$ customers:
    \begin{itemize}
      \item $\overline{11a} = 115$ have purchased fruits.
      \item $\overline{9b} = 96$ have purchased vegetables.
      \item $\overline{8a} = 85$ have purchased bakery items.
      \item $\overline{4b} = 46$ have purchased both fruits and vegetables.
      \item $\overline{3b} = 36$ have purchased both vegetables and bakery items.
      \item $\overline{2a} = 25$ have purchased both fruits and bakery items.
      \item $\overline{1a} = 15$ have purchased all three categories.
    \end{itemize}
    \item Let:
    \begin{itemize}
      \item $U$ be the set of all customers.
      \item $A$ is the set of customers who bought fruits.
      \item $B$ is the set of customers who bought vegetables.
      \item $C$ is the set of customers who bought bakery items.
      \item $E$ is the set of customers who bought nothing.
    \end{itemize}
    \item Based on the given data, we refine the above data to support solving the problem in a simpler way by using the concepts of sets and we will use these facts to find the number of customers who bought nothing:
    \begin{itemize}
      \item $|U| = 356$
      \item $|A| = 115$
      \item $|B| = 96$
      \item $|C| = 85$
      \item $|A \cap B| = 46$
      \item $|B \cap C| = 36$
      \item $|A \cap C| = 25$
      \item $|A \cap B \cap C| = 15$
      \item $|E|$?
    \end{itemize}
  \end{itemize}
\subsubsection*{Formula:}
  We can get \textbf{the number of customers who bought nothing ($|E|$)} by subtracting \textbf{the number of customers who bought at least one category ($|A \cup B \cup C|$)} from the \textbf{total number of customers ($|U|$)}, which means:
  \[
  |E| = |U| - |A \cup B \cup C| \tag{1}
  \]

  I will use the \textbf{principle of inclusion-exclusion} for three sets to determine the number of customers who bought nothing at the store, which is |E|, by finding $|A \cup B \cup C|$:
    \[
    |A \cup B \cup C| = |A| + |B|+|C| - |A \cap B| - |B \cap C| - |A \cap C| + |A \cap B \cap C| \tag{2}
    \]

\subsubsection*{Substitution:}
  Substitute the data we have in the given part into equation (2), we have:
  \begin{align*}
    (2) \Leftrightarrow |A \cup B \cup C| &= 115 + 96 + 85 - 46 - 36 - 25 + 15 \\
                      &= (115 + 96 + 85) - (46 + 36 + 25) + 15 \\
                      &= 296 - 107 + 15 \\
                      &= 189 + 15 \\
                      &= 204
  \end{align*}

  Substitute $|A \cup B \cup C|$ we just got above and $|U|$ into the equation (1) to find $|E|$, we have:
  \begin{align*}
    |E| &= |U| - |A \cup B \cup C| \\
        &= 356 - 204  \\
        &= 152
  \end{align*}

\subsubsection*{Verification:}
To confirm the union calculation step-by-step:
\begin{itemize}
  \item Sum of individual sets: $115 + 96 + 85 = 296$
  \item Subtract pairwise intersections: $296 - 46 - 36 - 25 = 296 - 107 = 189$
  \item Add back the triple intersection: $189 + 15 = 204$
\end{itemize}
This matches the result I did above, so the number of customers who purchased nothing is indeed 152.
\subsubsection*{Conclusion:}
The number of customers who have not purchased anything is \boxed{152}.
\section{Determine the Cardinality of a Given Bag (Multiset) (P2)}
\label{sec:P2}
\subsection{List the bag of prime factors for each of the provided numbers}
\subsubsection*{The Problem:}
List the bag of prime factors for each of the provided numbers.
  \begin{itemize}
    \item $\overline{1a2}$
    \item $\overline{2b0}$
  \end{itemize}


\subsubsection*{Given:}
\subsubsection*{Formula:}
\subsubsection*{Substitution:}
\subsubsection*{Verification:}
\subsubsection*{Conclusion:}

\subsection{Find the cardinalities}

\subsubsection*{The Problem:}
Find the cardinalities of
  \begin{itemize}
    \item each of the bags above.
    \item the intersection of the bags above.
    \item the union of the bags above.
    \item the difference of the bags above.
  \end{itemize}

\subsubsection*{Given:}
\subsubsection*{Formula:}
\subsubsection*{Substitution:}
\subsubsection*{Verification:}
\subsubsection*{Conclusion:}

\subsection{List the bag of prime factors for each of the provided numbers}

\subsubsection*{The Problem:}
List the bag of prime factors for each of the provided numbers:
\begin{itemize}
  \item \(\overline{1a2} = \overline{152}\)
  \item \(\overline{2b0} = \overline{260}\)
\end{itemize}

\subsubsection*{Given:}
Using \( a = 5 \) and \( b = 6 \) from the ID (BD00536), we have:
\begin{itemize}
  \item \(\overline{1a2} = 152\), since \( a = 5 \).
  \item \(\overline{2b0} = 260\), since \( b = 6 \).
\end{itemize}
We need to find the prime factorization of each number and represent them as bags (multisets), where a bag allows repeated elements, and the multiplicity of each prime factor is the number of times it appears in the factorization.

\subsubsection*{Formula:}
To find the bag of prime factors for a number, we perform prime factorization by dividing the number by the smallest possible prime numbers repeatedly until the quotient is 1 \cite{burton2010}. The bag contains all prime factors, including repetitions, based on their multiplicity in the factorization.

\subsubsection*{Substitution:}
Let’s compute the prime factorization for each number.

\textbf{For \(\overline{1a2} = 152\):}
\begin{itemize}
  \item Divide by 2: \( 152 \div 2 = 76 \)
  \item Divide by 2: \( 76 \div 2 = 38 \)
  \item Divide by 2: \( 38 \div 2 = 19 \)
  \item 19 is a prime number.
\end{itemize}
Thus, the prime factorization of 152 is:
\[
152 = 2^3 \cdot 19
\]
The bag of prime factors for 152 is:
\[
B_1 = \{2, 2, 2, 19\}
\]

\textbf{For \(\overline{2b0} = 260\):}
\begin{itemize}
  \item Divide by 2: \( 260 \div 2 = 130 \)
  \item Divide by 2: \( 130 \div 2 = 65 \)
  \item Divide by 5: \( 65 \div 5 = 13 \)
  \item 13 is a prime number.
\end{itemize}
Thus, the prime factorization of 260 is:
\[
260 = 2^2 \cdot 5 \cdot 13
\]
The bag of prime factors for 260 is:
\[
B_2 = \{2, 2, 5, 13\}
\]

\subsubsection*{Verification:}
To verify:
\begin{itemize}
  \item For 152: \( 2 \cdot 2 \cdot 2 \cdot 19 = 8 \cdot 19 = 152 \). Correct.
  \item For 260: \( 2 \cdot 2 \cdot 5 \cdot 13 = 4 \cdot 5 \cdot 13 = 20 \cdot 13 = 260 \). Correct.
\end{itemize}
The bags are:
\begin{itemize}
  \item \( B_1 = \{2, 2, 2, 19\} \)
  \item \( B_2 = \{2, 2, 5, 13\} \)
\end{itemize}

\subsubsection*{Conclusion:}
The bags of prime factors are:
\begin{itemize}
  \item For \(\overline{1a2} = 152\): \( \{2, 2, 2, 19\} \)
  \item For \(\overline{2b0} = 260\): \( \{2, 2, 5, 13\} \)
\end{itemize}

\subsection{Find the cardinalities}

\subsubsection*{The Problem:}
Find the cardinalities of:
\begin{itemize}
  \item Each of the bags above (\( B_1 \) and \( B_2 \)).
  \item The intersection of the bags (\( B_1 \cap B_2 \)).
  \item The union of the bags (\( B_1 \cup B_2 \)).
  \item The difference of the bags (\( B_1 \setminus B_2 \)).
\end{itemize}

\subsubsection*{Given:}
From the previous subsection:
\begin{itemize}
  \item \( B_1 = \{2, 2, 2, 19\} \)
  \item \( B_2 = \{2, 2, 5, 13\} \)
\end{itemize}

\subsubsection*{Formula:}
For multisets (bags), as defined in \cite{epp2020}:
\begin{itemize}
  \item The \textbf{cardinality} of a bag is the total number of elements, counting repetitions.
  \item The \textbf{intersection} \( B_1 \cap B_2 \) contains the elements common to both bags, with the multiplicity of each element being the minimum of its multiplicities in \( B_1 \) and \( B_2 \).
  \item The \textbf{union} \( B_1 \cup B_2 \) contains all elements from both bags, with the multiplicity of each element being the maximum of its multiplicities in \( B_1 \) and \( B_2 \).
  \item The \textbf{difference} \( B_1 \setminus B_2 \) contains elements in \( B_1 \) that remain after removing elements that appear in \( B_2 \), with multiplicities adjusted accordingly (if an element’s multiplicity in \( B_2 \) is less than in \( B_1 \), the difference includes the remaining occurrences).
  \item The cardinality of a bag \( B \), denoted \( |B| \), is the sum of the multiplicities of its distinct elements.
\end{itemize}
Additionally, the cardinality of the union can be verified using the inclusion-exclusion principle for multisets \cite{rosen2019}:
\[
|B_1 \cup B_2| = |B_1| + |B_2| - |B_1 \cap B_2|
\]

\subsubsection*{Substitution:}
Let’s compute each part.

\textbf{Cardinality of each bag:}
\begin{itemize}
  \item For \( B_1 = \{2, 2, 2, 19\} \):
    \[
    |B_1| = 3 \text{ (for the three 2’s)} + 1 \text{ (for the one 19)} = 4
    \]
  \item For \( B_2 = \{2, 2, 5, 13\} \):
    \[
    |B_2| = 2 \text{ (for the two 2’s)} + 1 \text{ (for the one 5)} + 1 \text{ (for the one 13)} = 4
    \]
\end{itemize}

\textbf{Intersection \( B_1 \cap B_2 \):}
To find the intersection, we take the common elements with the minimum multiplicity for each:
\begin{itemize}
  \item Element 2: Multiplicity in \( B_1 \) is 3, in \( B_2 \) is 2. Minimum is 2. Include \( \{2, 2\} \).
  \item Element 19: Multiplicity in \( B_1 \) is 1, in \( B_2 \) is 0. Minimum is 0. Exclude 19.
  \item Element 5: Multiplicity in \( B_1 \) is 0, in \( B_2 \) is 1. Minimum is 0. Exclude 5.
  \item Element 13: Multiplicity in \( B_1 \) is 0, in \( B_2 \) is 1. Minimum is 0. Exclude 13.
\end{itemize}
Thus:
\[
B_1 \cap B_2 = \{2, 2\}
\]
Cardinality:
\[
|B_1 \cap B_2| = 2
\]

\textbf{Union \( B_1 \cup B_2 \):}
For the union, we take all elements with the maximum multiplicity for each:
\begin{itemize}
  \item Element 2: Multiplicity in \( B_1 \) is 3, in \( B_2 \) is 2. Maximum is 3. Include \( \{2, 2, 2\} \).
  \item Element 19: Multiplicity in \( B_1 \) is 1, in \( B_2 \) is 0. Maximum is 1. Include \( \{19\} \).
  \item Element 5: Multiplicity in \( B_1 \) is 0, in \( B_2 \) is 1. Maximum is 1. Include \( \{5\} \).
  \item Element 13: Multiplicity in \( B_1 \) is 0, in \( B_2 \) is 1. Maximum is 1. Include \( \{13\} \).
\end{itemize}
Thus:
\[
B_1 \cup B_2 = \{2, 2, 2, 19, 5, 13\}
\]
Cardinality:
\[
|B_1 \cup B_2| = 3 \text{ (for 2)} + 1 \text{ (for 19)} + 1 \text{ (for 5)} + 1 \text{ (for 13)} = 6
\]

\textbf{Difference \( B_1 \setminus B_2 \):}
For the difference, we remove from \( B_1 \) the elements that appear in \( B_2 \), respecting multiplicities:
\begin{itemize}
  \item Element 2: Multiplicity in \( B_1 \) is 3, in \( B_2 \) is 2. After removing two 2’s, \( 3 - 2 = 1 \). Include \( \{2\} \).
  \item Element 19: Multiplicity in \( B_1 \) is 1, in \( B_2 \) is 0. Remains 1. Include \( \{19\} \).
  \item Elements 5 and 13: Not in \( B_1 \), so they don’t affect the difference.
\end{itemize}
Thus:
\[
B_1 \setminus B_2 = \{2, 19\}
\]
Cardinality:
\[
|B_1 \setminus B_2| = 1 \text{ (for 2)} + 1 \text{ (for 19)} = 2
\]

\subsubsection*{Verification:}
To verify:
\begin{itemize}
  \item Cardinality of \( B_1 \): Counts \( \{2, 2, 2, 19\} \), which has 4 elements. Correct.
  \item Cardinality of \( B_2 \): Counts \( \{2, 2, 5, 13\} \), which has 4 elements. Correct.
  \item Intersection: \( \{2, 2\} \) has 2 elements. Matches the minimum multiplicities.
  \item Union: \( \{2, 2, 2, 19, 5, 13\} \) has 6 elements. Matches the maximum multiplicities.
  \item Difference: Removing two 2’s from \( B_1 \) leaves one 2 and one 19, so \( \{2, 19\} \) has 2 elements. Correct.
  \item Using the inclusion-exclusion principle \cite{rosen2019}:
    \[
    |B_1 \cup B_2| = |B_1| + |B_2| - |B_1 \cap B_2|
    \]
    \[
    6 = 4 + 4 - 2 = 6
    \]
    This confirms the union’s cardinality.
\end{itemize}

\subsubsection*{Conclusion:}
The cardinalities are:
\begin{itemize}
  \item Cardinality of \( B_1 = \{2, 2, 2, 19\} \): \( |B_1| = 4 \)
  \item Cardinality of \( B_2 = \{2, 2, 5, 13\} \): \( |B_2| = 4 \)
  \item Cardinality of the intersection \( B_1 \cap B_2 = \{2, 2\} \): \( |B_1 \cap B_2| = 2 \)
  \item Cardinality of the union \( B_1 \cup B_2 = \{2, 2, 2, 19, 5, 13\} \): \( |B_1 \cup B_2| = 6 \)
  \item Cardinality of the difference \( B_1 \setminus B_2 = \{2, 19\} \): \( |B_1 \setminus B_2| = 2 \)
\end{itemize}
\[
\boxed{
\begin{array}{l}
|B_1| = 4, \\
|B_2| = 4, \\
|B_1 \cap B_2| = 2, \\
|B_1 \cup B_2| = 6, \\
|B_1 \setminus B_2| = 2
\end{array}
}
\]

\section{Determine the Inverse of a Function Using Appropriate Mathematical Techniques (M1)}

\label{sec:M1}

\section{Formulate Corresponding Proof Principles to Prove Properties about Defined Sets (D1)}

\label{sec:D1}

\chapter{Analyze Mathematical Structures of Objects Using Graph Theory (LO2)}
\label{chap:LO2}

\section{Model Contextualized Problems Using Trees, both Quantitatively and Qualitatively (P3)}
\label{sec:P3}


\section{Use Dijkstra’s Algorithm to Find a Shortest Path Spanning Tree in Graph (P4)}
\label{sec:P4}

\section{Assess whether an Eulerian and Hamiltonian Circuit Exists in an Undirected Graph (M2)}
\label{sec:M2}

\section{Construct a Proof of the Five-Color Theorem (D2)}
\label{sec:D2}

\chapter{Investigate Solutions to Problem Situations Using the Application of Boolean Algebra (LO3)}
\label{chap:LO3}

\section{Diagram a Binary Problem in the Application of Boolean Algebra (P5)}
\label{sec:P5}

\section{Produce a Truth Table and Its Corresponding Boolean Equation from an Applicable Scenario (P6)}
\label{sec:P6}

\section{Simplify a Boolean Equation Using Algebraic Methods (M3)}
\label{sec:M3}

\section{Design a Complex System Using Logic Gates (D1)}
\label{sec:D1}

\chapter{Explore Applicable Concepts within Abstract Algebra (LO4)}
\label{chap:LO4}

\section{Describe the Distinguishing Characteristics of Different Binary Operations that are Performed on the same Set (P7)}
\label{sec:P7}

\section{Determine the Order of a Group and the Order of a Subgroup in Given Examples (P8)}
\label{sec:P8}

\section{Validate whether a Given Set with a Binary Operation is indeed a Group (M4)}
\label{sec:M4}

\section{Explore, with the Aid of a Prepared Presentation, the Application of Group Theory Relevant to your Given Example (D4)}
\label{sec:D4}



\newpage
\section*{Conclusion}
\newpage
\section*{Evaluation}
\bibliography{references}

\end{document}

% Run as following order:
% pdflatex main 
% bibtex main
% pdflatex main
% pdflatex main


